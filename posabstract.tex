\phantomsection
\stepcounter{articleid}
\addcontentsline{toc}{subsection}{Lücking et~al. Modelling and evaluation of the snow and ice melt in the Großglockner Region}
\begin{flushleft}

\abstrtitle{Modelling and evaluation of the snow and ice melt in the Großglockner Region}

\name{Sophie Lücking$^1$, Franziska Koch$^1$, Herbert Formayer$^1$}

\index{Lücking, Sophie|textit}
\index{Koch, Franziska}
\index{Formayer, Herbert}
\institute{$^1$ Institute of Hydrology and Water Management, BOKU Vienna, Austria}

\email{sophie.luecking@boku.ac.at}

\end{flushleft}

\noindent


The cryosphere in high alpine regions is dramatically affected by climate change. All over the world people are dependent on the melt water of the cryosphere as they use it for drinking water, hydropower and agriculture. It is an important task to simulate the processes in high alpine regions. This study features an adapted version of the GR4J model, which is a simple conceptual rainfall-runoff model, to simulate snow and ice. The aim of this thesis is to test this new adaptation in a glaciated catchment in Austria for the time period 1986–2022.

The results calibrated with the discharge show the best Nash-Sutcliffe Efficiency value (0.86), when dividing the catchment into elevation bands. The validation of the simulated snow extent with the snow detected and the georeferenced webcam pictures reveal an R$^2$ of 0.92. The results indicate that the yearly discharge consists on average of 37\% ice melt and 33\% snow melt. A focus on glacier melt reveals an increase in ice melt at higher elevations as well a reduction of ice melt in lower elevations
due to glacier losses.
\newpage{}
\phantomsection
\stepcounter{articleid}
\addcontentsline{toc}{subsection}{Valent and Las. Quantifying projected effects of climate change on low-flows in the Salzach River, Austria: An approach for adapting stochastic weather generator outputs}
\begin{flushleft}

\abstrtitle{Quantifying projected effects of climate change on low-flows in the Salzach River, Austria: An approach for adapting stochastic weather generator outputs}

\name{Peter Valent$^1$, Marcin Las$^1$}

\index{Valent, Peter|textit}
\index{Las, Marcin}
\institute{$^1$ Institute of Hydraulic and Water Resources Engineering, Vienna University of Technology, Karlsplatz 13/222, A-1040 Wien, Vienna, Austria}

\email{valent@hydro.tuwien.ac.at}

\end{flushleft}

\noindent

In the context of ongoing climate change, the use of climate model predictions plays a crucial role in preparing adaptation strategies that could reflect the expected changes in low-flow conditions within water systems. These predictions are often non-stationary and thus not very suitable for a comprehensive statistical analysis of extremes, mainly due to very short periods of quasi-stationary conditions. In this study, we address these problems by adapting a 10,000-year long dataset of synthetic meteorological outputs, representing the current climate conditions and produced by a large multi-site stochastic weather generator, to account for both spatial and temporal variability of the changing climate through a delta-change approach. We adapted this dataset to depict the mean climate conditions of the final three decades of the 21st century, as defined by four different scenarios (2xGCM, 2xRCP) incorporated into the high-resolution Austrian National Climate dataset (ÖKS15). We employed an HBV-distributed hydrological model to simulate synthetic river discharges for both the current and four future climate conditions in the Salzach River, Austria. Our subsequent analysis of the low-flows reveals that the more extreme climate conditions, as characterized by increased temperatures and rainfall amounts, also transfer into larger low-flows and shorter dry periods compared to the current climate. The results can be attributed to two factors: (1) the low-flow season in the Salzach River predominantly occurs during the winter, leading to reduced snow accumulation in the catchment due to rising temperatures, and (2) the adopted delta change approach reflects the changes in rainfall amounts but does not address the changes in rainfall frequencies. Future research should, therefore, prioritize the development of methods that can accommodate qualitative changes in rainfall distribution.

\includepdf[addtotoc={1, subsubsection, 1, Poster,P02}]{Posters/P02.pdf}

\newpage{}
\phantomsection
\stepcounter{articleid}
\addcontentsline{toc}{subsection}{Chappon and Bene. Predicting water level changes in Lake Velence – an invitation to participate in a scientific challenge}
\begin{flushleft}

\abstrtitle{Predicting water level changes in Lake Velence – an invitation to participate in a scientific challenge}

\name{Máté Chappon$^1$, Katalin Bene$^1$}

\index{Chappon, Máté|textit}
\index{Bene, Katalin}
\institute{$^1$ National Laboratory for Water Science and Water Security, Széchenyi István University, Department of Transport Infrastructure and Water Resources Engineering  Egyetem tér 1. H-9026 Győr, Hungary }

\email{chappon.mate@sze.hu}

\end{flushleft}

\noindent

The prediction of lake water levels is becoming a crucial task of water resources management as changes in the climate and land use, coupled with increased water use, cause the water level of lakes and reservoirs to decrease progressively in many locations around the globe. Recently, an extreme drought caused record low water levels in Lake Velence, Hungary, that undermined recreational tourism around the lake, thereby triggering conflicts between water users within the catchment. Water managers' efforts to find solutions require reliable lake water level predictions.

This poster invites researchers, engineers, and enthusiasts to participate in a scientific challenge of predicting the water levels of Lake Velence at the beginning of each month during 2024. To participate in the challenge, one must estimate the daily water levels for the upcoming month. 

Past data for the 2002-2022 period will be provided, including catchment characteristics and the operational rules of the hydraulic structures, stage and flow measurement in the catchment area, and the lake's current official water budget calculation method. Considering the known uncertainties of the current calculation method, participants are encouraged to develop their own unique method to determine lake water levels. Initial water levels and meteorological forecasts for the basin area will be given at the beginning of each month. 

The monthly winner is the one whose prediction results in the lowest root mean squared error. The overall champion of the challenge is the one with the most monthly wins throughout the year.

\includepdf[addtotoc={1, subsubsection, 1, Poster,P03}]{Posters/P03.pdf}

\newpage{}
\phantomsection
\stepcounter{articleid}
\addcontentsline{toc}{subsection}{Psilovikos et~al. Land Use Land Change due to Wildfires: Impacts on Surface Runoff and Erosion. The Case Study of Mt. Pelion, Thessaly, Greece}
\begin{flushleft}

\abstrtitle{Land Use Land Change due to Wildfires: Impacts on Surface Runoff and Erosion. The Case Study of Mt. Pelion, Thessaly, Greece}

\name{Aris Psilovikos$^1$, Theodoros Papathanasiou$^1$, Georgios Mpouras$^1$, Dimitrios Malamataris$^1$, Thomas Psilovikos$^3$, Anthimos Spiridis$^2 $}

\index{Psilovikos, Aris|textit}
\index{Papathanasiou, Theodoros}
\index{Mpouras, Georgios}
\index{Malamataris, Dimitrios}
\index{Psilovikos, Thomas}
\index{Spiridis, Anthimos}
\institute{$^1$ Laboratory of Ecohydrology \& Inland Water Management, Department of Ichthyology and Aquatic Environment, University of Thessaly, Fytoko Street, 38446, N. Ionia Magnisias, Greece}

\institute{$^2$ HYETOS S.A. Consulting Company, Ippodromiou Sq. 7, 54622, Thessaloniki, Greece }

\institute{$^3$ Laboratory of Forest Engineering and Topography, School of Forestry and Natural Environment, Aristotle University of Thessaloniki, P.O. Box 226, 54124, Thessaloniki, Greece. }

\email{psiloviko@uth.gr}

\end{flushleft}

\noindent

Wildfires can trigger dramatic increases in surface runoff and erosion, because of the burned vegetation and the appearance of a condition of soil-water repellence. Fire-enhanced surface runoff generation and soil erosion, constitute adverse effects of high concern for a long-term future period after a fire event occurrence. 

The current study, investigates the increase on peak surface runoff and sediment loss as a result of a fire event that occurred in Mountain Pelion area in Greece on June 27$^{\textrm{th}}$, 2007. Mountain Pelion is located in Magnesia Prefecture (Thessaly Region, Central Greece), between the Aegean Sea to the northeast and the Pagasitikos Gulf to the southwest and its capital is the city of Volos. The boundaries of the burned area were determined using both a satellite image, as well as, a relative map produced by the Regional Forest Protection Service and the total burned area was found to be almost 60 km$^2$. 

The land cover of the area prior to fire event, mainly included forest, seminatural and agricultural areas as determined by the raster datasets produced by the Copernicus Land Cover program. The change of peak surface runoff discharge pre- and post- fire event, was estimated using the Natural Resources Conservation Service – CN (NRCS-CN) method, while the sediment loss was also estimated, using the Stiny – Herheulidze Method. The fire event was found to significantly increased peak surface runoff, which may cause the occurrence of flood events in the downstream area. 

The results of the current study demonstrated that fire-induced alterations to the soil structure as reflected by the increase of the Curve Number parameter, cause an increase in runoff volume and sediment yield. Post-fire runoff and soil losses calculation, constitutes a crucial procedure for the better adaptation of burned areas to post – fire flood events. 

Keywords: Wildfires; Land Use – Land Cover Change; Surface Runoff; NRCS-CN Method; Erosion, Stiny – Herheulidze Method, Sediment Yield.

\includepdf[pages=-,pagecommand={}, nup=1x2, width=\textwidth, delta=0 20mm, addtotoc={1, subsubsection, 1, Presentation,Pe04}]{Presis/P04Psilovikos.pdf}

\includepdf[addtotoc={1, subsubsection, 1, Poster,P04}]{Posters/P04.pdf}

\newpage{}
\phantomsection
\stepcounter{articleid}
\addcontentsline{toc}{subsection}{Basso et~al. The effect of multiple reservoirs on flood peak reduction: the case of two Austrian catchments}
\begin{flushleft}

\abstrtitle{The effect of multiple reservoirs on flood peak reduction: the case of two Austrian catchments}

\name{Anna Basso$^2$, Miriam Bertola$^1$, Peter Valent$^1$, Alberto Viglione$^2$, Günter Blöschl$^1$}

\index{Basso, Anna|textit}
\index{Bertola, Miriam}
\index{Valent, Peter}
\index{Viglione, Alberto}
\index{Blöschl, Günter}
\institute{$^1$ Institute of Hydraulic Engineering and Water Resources Management, Vienna University of Technology}

\institute{$^2$ Department of Environment, Land and Infrastructure Engineering, Polytechnic University of Turin}

\email{s300922@studenti.polito.it}

\end{flushleft}

\noindent

In Austria there are 78 large reservoirs, and in most cases, they are located in nested catchments. The presence of a reservoir generally produces an attenuation effect on flood peaks downstream; however, the combined effect of multiple reservoirs in a catchment remains unclear. 

The aim of this work is to analyze the combined effect of multiple dams on flood peaks along the river network on a regional scale. Its evolution can be studied in space, considering how the effect changes along the river segments and as a function of the return period of the flood event.

The method used consists of two steps: (i) the design hydrographs are evaluated at several locations along the river network for 30, 100 and 300 year return periods, based on interpolated flood quantiles in ungauged catchments; (ii) the peak reduction is estimated based on the information on dams in the catchment (i.e., the number, position, storage capacity, and drainage area) and, using the concept of an equivalent reservoir, the protection ratio and the filling discharge. 

The effect of the reservoirs is evaluated here as a percentage of the reduction of the flood wave peak. Preliminary results have been obtained for the Salzach and the Austrian parts of the Drau catchments that show that the relative peak reduction is highest near the reservoirs and rapidly diminishes downstream for each return period and different combinations of the dams. 
\newpage{}
\phantomsection
\stepcounter{articleid}
\addcontentsline{toc}{subsection}{Pulka et~al. Evaluation of precipitation corrections to improve reservoir inflow predictions by a conceptual hydrological model in the Malta Valley, Austria}
\begin{flushleft}

\abstrtitle{Evaluation of precipitation corrections to improve reservoir inflow predictions by a conceptual hydrological model in the Malta Valley, Austria}

\name{Thomas Pulka$^1$, Mathew Herrnegger$^1$, Caroline Ehrendorfer$^1$, Sophie Lücking$^1$, Karsten Schulz$^1$, Franziska Koch$^1$}

\index{Pulka, Thomas|textit}
\index{Herrnegger, Mathew}
\index{Ehrendorfer, Caroline}
\index{Lücking, Sophie}
\index{Schulz, Karsten}
\index{Koch, Franziska}
\institute{$^1$ Institute of Hydrology and Water Management, BOKU Vienna, Austria}

\email{Thomas.pulka@boku.ac.at}

\end{flushleft}

\noindent

Conceptual hydrological models are widely used for the prediction of high-alpine runoff, which is essential for efficient reservoir management in hydropower production. The performance of these models is, however, strongly linked to the quality of the meteorological forcing data. The inherent uncertainties of distributed meteorological datasets in high-alpine areas are especially pronounced for precipitation. We evaluated different precipitation corrections to downscaled INCA data serving as inputs for hydrological modelling. The physically-based Alpine3D snowpack model, in combination with satellite-based snow depth maps, was used to derive quantitatively and spatially distributed precipitation scaling for the winter period; and a stepwise linear correction model was applied for the correction of the summer precipitation. Using the corrected precipitation, the Kölnbrein reservoir inflow (Malta Valley, Austria) was simulated with the COSERO precipitation-runoff model on an hourly timescale for the period 2015 – 2023. Different combinations of the winter and summer corrections were evaluated according to their potential to improve hydrological modelling. The preliminary results show that the combined correction of the winter and summer precipitation leads to the greatest improvements overall and during the snowmelt season.

Acknowledgements: We thank the VERBUND Energy4Business GmbH for fruitful discussions and providing us with data.
\newpage{}
\phantomsection
\stepcounter{articleid}
\addcontentsline{toc}{subsection}{Négyesi and Nagy. Rainfall-runoff modeling based on an Adaptive Neuro-Fuzzy Inference System in Hungary}
\begin{flushleft}

\abstrtitle{Rainfall-runoff modeling based on an Adaptive Neuro-Fuzzy Inference System in Hungary}

\name{Klaudia Négyesi$^1$, Eszter D. Nagy$^1$}

\index{Negyesi, Klaudia@N\'egyesi, Klaudia|textit}
\index{Nagy, Eszter}
\institute{$^1$ Department of Hydraulic and Water Resources Engineering, Faculty of Civil Engineering, Budapest University of Technology and Economics}

\email{negyesiklaudia@edu.bme.hu}

\end{flushleft}

\noindent

Artificial neural networks (ANNs) have become one of the most promising tools to simulate complex phenomena in various fields. In hydrology, the first ANN-based rainfall-runoff models appeared in the 1990s. Since then, the ANN-based models have shown results comparable to conceptual models. According to the most recent publications, the Adaptive Neuro-Fuzzy Inference System (ANFIS) is one of the best-performing hydrological data-driven models. It can model nonlinear functions and identify nonlinear components while combining the flexibility of fuzzy modeling with the learning capability of neural networks. The present study is the first step of our detailed research. It reviews the international literature to identify the possibilities of rainfall-runoff modeling using ANFIS. Furthermore, an analysis of its applicability through model simulations was performed in MATLAB for the Hungarian catchments. The available datasets and the best input data combinations were examined. The results provide an insight into the limitations and potential of applying ANFIS-based rainfall-runoff models in Hungary.  



\newpage{}
\phantomsection
\stepcounter{articleid}
\addcontentsline{toc}{subsection}{Khalil and Parajka. Sparse runoff observations ample enough to improve multiple-objective calibration of satellite data of snow cover and soil moisture (preliminary results)}
\begin{flushleft}

\abstrtitle{Sparse runoff observations ample enough to improve multiple-objective calibration of satellite data of snow cover and soil moisture (preliminary results)}

\name{Asma Khalil$^1$, Juraj Parajka$^1 $}

\index{Khalil, Asma|textit}
\index{Parajka, Juraj}
\institute{$^1$ Institute of Hydraulic Engineering and Water Resources Management, TU Wien, Vienna, Austria}

\email{khalil@hydro.tuwien.ac.at}

\end{flushleft}

\noindent

Remote sensing has great potential for setting hydrological models and hence predicting runoff hydrographs in regions with sparse ground observations. The aim of this study was to analyze the annual and seasonal patterns of agreement between the model simulations and observations and to propose and evaluate strategies using only irregular runoff observations to constrain a conceptual hydrological model. In the first step, the model was calibrated to satellite images of the snow cover and soil moisture (without using runoff data). In the second step, the seasonal and annual agreement between the runoff model simulations and observations is analyzed in a large number of Austrian catchments. Identification of the spatial and temporal patterns of the agreement serves for the development of different scenarios of adding sparse (irregular) runoff observations into the multiple-objective calibration. The preliminary result of this study derives the patterns in the agreement and disagreement of the observed and simulated runoff, i.e., when and where the satellite data allows setting the hydrological model and where and when some additional runoff data are needed along with a comparison of different strategies to constrain the conceptual hydrological model.
\newpage{}
\phantomsection
\stepcounter{articleid}
\addcontentsline{toc}{subsection}{Rattayová et~al. Changes of the balance evapotranspiration trends in Austria}
\begin{flushleft}

\abstrtitle{Changes of the balance evapotranspiration trends in Austria}

\name{Viera Rattayová$^1$, Juraj Parajka$^2$, Kamila Hlavčová$^1$}

\index{Rattayová, Viera|textit}
\index{Parajka, Juraj}
\index{Hlavčová, Kamila}
\institute{$^1$ Department of Land and Water Resources Management, Faculty of Civil Engineering Slovak University of Technology in Bratislava, Radlinského 11, 813 68 Bratislava, Slovakia}

\institute{$^2$ Institute of Hydraulic Engineering and Water Resources Management, Vienna University of Technology, Vienna, Austria.}

\email{viera.rattayova@stuba.sk}

\end{flushleft}

\noindent

Evapotranspiration plays an essential role in the hydrological cycle; however, this water cycle component is difficult to quantify. Changes in evapotranspiration are a consequence of climate change, and their magnitude is affected by changes in the available water and energy. Describing changes in evapotranspiration provides an important tool for decision-making processes in water resource management. This paper seeks to investigate how the water balance evapotranspiration trends have changed over time. We calculated evapotranspiration from the SPARTACUS database of daily runoff and precipitation totals for 140 watersheds in Austria. Trend analyses were realized with the Mann-Kendall test of trends and modifications of the Man-Kendall test of trends and Sen’s slope in two periods, i.e., 1981–2020 and 2001-2020. The results show changes in the water balance evapotranspiration trends between the two examined periods. In the last years (2001–2020), the occurrence of increasing trend in water balance evapotranspiration of many watersheds decreased, there were also detected increasing occurrence of watersheds with decreasing trend in this period. Significant trends were mostly detected in watersheds with mean altitude lower than 500 m a.s.l. for both periods. 
\newpage{}
\phantomsection
\stepcounter{articleid}
\addcontentsline{toc}{subsection}{Sabová and Kohnová. Pooling of the long-term average monthly runoff using the PCA method and K-means clustering in Slovakia}
\begin{flushleft}

\abstrtitle{Pooling of the long-term average monthly runoff using the PCA method and K-means clustering in Slovakia}

\name{Zuzana Sabová$^1$, Silvia Kohnová$^1$}

\index{Sabová, Zuzana|textit}
\index{Kohnová, Silvia}
\institute{$^1$ Slovak University of Technology in Bratislava, Slovakia}

\email{zuzana.sabova@stuba.sk}

\end{flushleft}

\noindent

This study is aimed at creating a new pooling scheme for the long-term average monthly runoff regime on the territory of Slovakia. For the analysis, the data of the long-term average monthly discharges from 57 gauging stations of the whole territory of Slovakia, which was provided by the Slovak Hydrometeorological Institute, were used. The selected basin areas range from 7.25 km$^2$ (5130 – the Spariská gauging station on the Vydrica stream) to 11474.30 km$^2$ (9670 – the Streda nad Bodrogom gauging station on the Bodrog stream). The monthly discharge data available were from gauging stations for the new reference period from 1991 to 2020 according to the World Meteorological Organization (WMO). The input data were processed as normalised long-term average monthly discharge data. The appropriate number of clusters was determined according to the statistical analysis using the average Silhouette Width and the Elbow method. Subsequently, the PCA method and K-means clustering were performed to pool the catchments into groups.

The results present the outputs of the particular runoff regime in the selected gauging stations divided into five clusters. Cluster 1 is characterised by south-central Slovakia and central Slovakia; Cluster 2 by the northwest and northeast of the country; Cluster 3 for the centre of northern Slovakia; Cluster 4 for central Slovakia, and Cluster 5 for the east, south and west of Slovakia.

These results were compared with previous studies of monthly runoff regionalisation in Slovakia for the previous reference period of 1961-2000. Using the R Studio program, the most important characteristic features of the individual clusters of gauging stations created were also analysed, which could help incorporate other catchments into appropriate regional types in the future. The methodological procedure developed could also be used in further studies to predict future flow regime changes on the territory of Slovakia.

Key words: pooling types, K-means clustering, PCA, R Studio

Acknowledgements 
This study was supported by the Slovak Research and Development Agency under Contract No. APVV-20-0374 and VEGA Grant Agency No 1/0782/21. The authors thank the agencies for their research support.

\includepdf[addtotoc={1, subsubsection, 1, Poster,P10}]{Posters/P10.pdf}

\newpage{}
\phantomsection
\stepcounter{articleid}
\addcontentsline{toc}{subsection}{Széles et~al. Comparison of two hydrograph separation methods in the Hydrological Open Air Laboratory using stable isotopes}
\begin{flushleft}

\abstrtitle{Comparison of two hydrograph separation methods in the Hydrological Open Air Laboratory using stable isotopes}

\name{Borbála Széles$^1$, Ladislav Holko$^2$, Juraj Parajka$^1$, Stefan Wyhlidal$^3$, Katharina Schott$^4$, Christine Stumpp$^5$, Michael Stockinger$^5$, Patrick Hogan$^1$, Lovrenc Pavlin$^1$, Peter Strauss$^6$, Günter Blöschl$^1$}

\index{Széles, Borbála|textit}
\index{Holko, Ladislav}
\index{Parajka, Juraj}
\index{Wyhlidal, Stefan}
\index{Schott, Katharina}
\index{Stumpp, Christine}
\index{Stockinger, Michael}
\index{Hogan, Patrick}
\index{Pavlin, Lovrenc}
\index{Strauss, Peter}
\index{Blöschl, Günter}
\institute{$^1$ Institute of Hydraulic Engineering and Water Resources Management, Vienna University of Technology, Austria}

\institute{$^2$ Institute of Hydrology, Slovak Academy of Sciences, Slovakia}

\institute{$^3$ NES, Nuclear Engineering Seibersdorf GmbH, Austria}

\institute{$^4$ BOKU University of Natural Resources and Life Sciences, Institute of Soil Science, Stable Isotope Group, Austria}

\institute{$^5$ BOKU University of Natural Resources and Life Sciences, Institute for Soil Physics and Rural Water Management, Austria}

\institute{$^6$ Federal Agency of Water Management, Institute for Land and Water Management Research, Austria}

\email{szeles@hydro.tuwien.ac.at}

\end{flushleft}

\noindent

Exploring the isotopic composition of precipitation and streamflow in small catchments and the event and pre-event components of precipitation events using two-component and ensemble isotopic hydrograph separation may better explain overall catchment behaviour, more specifically, the origin of water. The aim of this study was to investigate the variability in the isotopic composition of precipitation and runoff in the 66 ha agricultural catchment in Austria known as the Hydrological Open Air Laboratory (HOAL) in order to compare two isotope hydrograph separation methods. Two-component (IHS) and ensemble (EIHS) isotopic hydrograph separations (for both 18O and 2H) were conducted for the catchment outlet for multiple events during the warm periods of 2013–2018. The results showed that the isotopic composition of the discharge in the HOAL remained nearly constant between the events which can be explained by the large diffuse groundwater discharge into the stream in the catchment. It was found that EIHS provided average new water fractions that were close to the averages obtained by IHS. Even though EIHS could not provide the ranges of the new water fractions during peak flows, this study proved that the results obtained by EIHS were close to the ones by IHS.
\newpage{}
\phantomsection
\stepcounter{articleid}
\addcontentsline{toc}{subsection}{Leivadiotis and Kohnová. Evaluating flood events caused by Medicane “Daniel” of the Thessaly district (Central Greece) using remote sensing data and techniques}
\begin{flushleft}

\abstrtitle{Evaluating flood events caused by Medicane “Daniel” of the Thessaly district (Central Greece) using remote sensing data and techniques}

\name{Evangelos Leivadiotis$^1$, Silvia Kohnová$^1$}

\index{Leivadiotis, Evangelos|textit}
\index{Kohnová, Silvia}
\institute{$^1$ Department of Land and Water Resources Management, Faculty of Civil Engineering, Slovak University of Technology in Bratislava, (Radlinskeho 11), Bratislava 81005, Bratislava}

\email{evangelos.leivadiotis@stuba.sk}

\end{flushleft}

\noindent

On September 4, 2023, Thessaly, Greece, witnessed a catastrophic flood due to Medicane Daniel. Extreme rainfall, ranging from 305 mm to 1096 mm between 4 and 12 September, caused extensive damage to the infrastructure, agriculture and houses. Seventeen casualties were recorded. This study is aimed at utilizing radar data from earth observation satellites to detect the inundated areas and evaluate the flood impacts. Sentinel images were acquired from https://scihub.copernicus.eu/ for 7, 10, 13, and 19 September, 2023, along with a CORINE Land use/Land cover map. Processing the satellite images involved several steps, including their calibration, noise removal, filtering, and polarization adjustment. Flooded areas were quantified for the specified dates, thereby revealing substantial coverage. The damage assessment focused on irrigated and non-irrigated land as well as pasture areas, highlighting the severity of the impact on them. The second phase incorporated the Flood Water Depth Estimation Tool (FwDET) to estimate the floodwater depths and revealed maximum depths near riverbanks. In conclusion, this study employed advanced remote sensing and GIS techniques to rapidly estimate flood characteristics and assess Medicane Daniel's damage to the Thessaly prefecture from September 4 to 12, 2023.

\includepdf[addtotoc={1, subsubsection, 1, Poster,P12}]{Posters/P12.pdf}

\newpage{}
\phantomsection
\stepcounter{articleid}
\addcontentsline{toc}{subsection}{Paulíková and Kohnová. Selection of the peak runoff coefficient estimation method for the Horné Orešany – Parná case study (Slovakia)}
\begin{flushleft}

\abstrtitle{Selection of the peak runoff coefficient estimation method for the Horné Orešany – Parná case study (Slovakia)}

\name{Lynda Paulíková$^1$, Silvia Kohnová$^1$}

\index{Paulíková, Lynda|textit}
\index{Kohnová, Silvia}
\institute{$^1$ Department of Land and Water Resources Management, Faculty of Civil Engineering, Slovak University of Technology in Bratislava, (Radlinskeho 11), Bratislava 81005, Bratislava Slovakia}

\email{lynda.paulikova@stuba.sk}

\end{flushleft}

\noindent

The research presents the results of the peak runoff coefficient estimated for the small Horné Orešany – Parná basin in Slovakia. The study aims to estimate the design peak runoff coefficient, which was determined by direct and indirect methods.

The first part of the study was devoted to calculating the design peak runoff coefficient according to the indirect method, which is based on the basic parameters for calculating runoff based on a rational formula. Three different critical durations of rain were used according to the calculation methods given by Nash (1960), Kirpich (1940) and Hrádek (1989). The critical rainfall was estimated for the nearest rain gauge station, i.e., Smolenice.

The second part of the study was devoted to the direct estimation of the design peak runoff coefficient. The analysis is based on the hourly flow measurements in the Horné Orešany – Parná basin for the 1989 – 2021 case study. For the analysis, we divided the annual maximal flows in the winter and summer. We separated and estimated the runoff volume and the time of rising and falling limbs from the selected peak flood waves. This was used to calculate the direct peak runoff coefficient and was subsequently subjected to statistical analysis according to the Johnson probability distribution. The results of the model are estimated peak runoff coefficient for a recurrence period of a hundred years.

The study results show that the direct method leads to lower estimates of the peak runoff coefficient in a small watershed (0.20 for the summer and 0.21 for the winter). On the other hand, the values determined by the indirect method are 0.18 – 0.39 in the winter months, but in the summer months, the estimated values are close to double, and in some cases, up to three times (0.42 – 0.90) those values, pointing to significant differences in the estimation of the peak runoff coefficient, which should be taken into account in future calculations.

The study was supported by the VEGA Grant No. 1/0782/21. The authors are very grateful for their research support.

Keywords: peak runoff coefficient; small basins; drain; rational formula

\includepdf[addtotoc={1, subsubsection, 1, Poster,P13}]{Posters/P13.pdf}

\newpage{}
\phantomsection
\stepcounter{articleid}
\addcontentsline{toc}{subsection}{Liová et~al. Constructing of control flood wave using a vine copula-based approach}
\begin{flushleft}

\abstrtitle{Constructing of control flood wave using a vine copula-based approach}

\name{Anna Liová$^1$, Roman Výleta$^1$, Kamila Hlavčová$^1$, Silvia Kohnová$^1$, Tomáš Bacigál$^2$, Ján Szolgay$^1$}

\index{Liová, Anna|textit}
\index{Výleta, Roman}
\index{Hlavčová, Kamila}
\index{Kohnová, Silvia}
\index{Bacigál, Tomáš}
\index{Szolgay, Ján}
\institute{$^1$ Department of Land and Water Resource Management, Slovak University of Technology, Bratislava, Slovakia}

\institute{$^2$ Department of Mathematics and Descriptive Geometry, Slovak University of Technology, Bratislava, Slovakia}

\email{anna.liova@stuba.sk}

\end{flushleft}

\noindent

To assess secure hydraulic structures we need to correctly estimate design values. In Slovakia in the framework of the safety assessment of dams during a flood load, each dam is required to be assessed for the critical load represented by a control flood wave. The control flood wave can sufficiently describe the impacts of flood events in many cases. In this study is proposed a methodology based on using empirical and statistical approaches for constructing synthetic design flood hydrographs. The method consists of a seasonal analysis of floods; the sampling of seasonal flood hydrographs; the dependence modelling of peaks, volumes, and durations using vine copulas; and determining the joint conditional return period of the flood wave.

The resulting set of hydrographs can further serve as the basis for simulations of the transmission of a control wave through a reservoir for different load conditions, the long-term loading of water structures by high-volume waves, or surge loads induced by flood waves with large peak discharges. This method allows the designer to select a hydrograph of different shapes, volumes, and durations for a selected design discharge with a known joint conditional probability of exceeding the volumes and durations for the flood risk analysis. The case study was carried out on the Parná river catchment in Slovakia.

\includepdf[pages=1-6,pagecommand={}, nup=1x2, width=\textwidth, delta=0 20mm, addtotoc={1, subsubsection, 1, Presentation,Pe14}]{Presis/P14Liova.pdf}

\includepdf[addtotoc={1, subsubsection, 1, Poster,P14}]{Posters/P14.pdf}

\newpage{}
\phantomsection
\stepcounter{articleid}
\addcontentsline{toc}{subsection}{Brolly et~al. 3D modelling of the hydrological observation site in the Hidegvíz-Valley from terrestrial laser scans}
\begin{flushleft}

\abstrtitle{3D modelling of the hydrological observation site in the Hidegvíz-Valley from terrestrial laser scans}

\name{Gábor Brolly$^1$, Noémi Ferenczi$^2$, Mátyás Mentes$^2$}

\index{Brolly, Gábor|textit}
\index{Ferenczi, Noémi}
\index{Mentes, Mátyás}
\institute{$^1$ University of Sopron, Faculty of Forestry, Inst. Geomatics and Civil Engineering}

\institute{$^2$ BSc student with the University of Sopron, Faculty of Forestry}

\email{gabor.brolly@uni-sopron.hu}

\end{flushleft}

\noindent

This study outlines the data acquisition and modelling of a hydro-meteorological observation site in the Hidegvíz Valley experimental catchment in West Hungary. The main goal is to record the spatial arrangement of the trees, site utilities, and the measuring equipment. A terrestrial laser scanner was applied to record high density point clouds from four station positions. Following the georeferencing of the individual scans and creation of a digital terrain model, the point measurements were classified into thematic categories by visual interpretation. The classified point cloud segments were used to create 3D surface and solid object models of the observation equipment as well as trees up to 2 meters in height. The resulting 3D models are appropriate for spatial documentation, and visualization, and they have potential to support the engineering design on the upgrade of the measuring equipment.

This presentation was made within the frame of the project TKP2021-NKTA-43 supported by the Ministry of Innovation and Technology of Hungary; National Research, Development and Innovation Fund.

\includepdf[addtotoc={1, subsubsection, 1, Poster,P15}]{Posters/P15Brolly.pdf}

\newpage{}
\phantomsection
\stepcounter{articleid}
\addcontentsline{toc}{subsection}{Muraközy et~al. Data integration and error analysis of the Botanical Garden’s hydrometeorological station in Sopron}
\begin{flushleft}

\abstrtitle{Data integration and error analysis of the Botanical Garden’s hydrometeorological station in Sopron}

\name{Lili Muraközy$^1$, Péter Kalicz$^1$, Márton Kiss$^2$, Zoltán Gribovszki$^1$}

\index{Muraközy, Lili|textit}
\index{Kalicz, Péter}
\index{Kiss, Márton}
\index{Gribovszki, Zoltán}
\institute{$^1$ Institute of Geomatics and Civil Engineering, Hydrology, University of Sopron, Hungary}

\institute{$^2$ Hungarian Meteorological Service, Sopron, Hungary}

\email{MurakozyL19@uni-sopron.hu}

\end{flushleft}

\noindent

The monitoring of hydrometeorological processes has been for centuries crucial for analysing the water balance of long-lived plant communities such as forests. In contrast, long-term measurements are crucial for evaluations of the impact of climate change. Meteorological records (which were the first regular examinations in Hungary) began in 1711 in Sopron. The Botanical Garden’s meteorological station at the University of Sopron operated from 1925.06.01. to 1974.04.24. and was the official station in Sopron at that time. The official station was moved to the Observatory of Kuruc Hill in 1974. The instruments that remained in the Botanical Garden continued to operate. The presentation reviews the integration of the station's data, the error analysis, and the simple data processing for hydrometeorological purposes.

The following joint projects (143972SNN project and the TKP2021-NKTA-43 project) supported the preparation of this paper. TKP2021-NKTA-43 has been implemented with the support provided by the Ministry of Innovation and Technology of Hungary (successor: Ministry of Culture and Innovation of Hungary) from the National Research, Development and Innovation Fund, financed under the TKP2021-NKTA funding scheme.

Supported by the ÚNKP-23-2-III-SOE-176 New National Excellence Program of the Ministry for Culture and Innovation from the source of the National Research, Development and Innovation Fund.

\includepdf[pages = -, pagecommand={}, addtotoc={1, subsubsection, 1, Short article,A16}]{ShortArt/P16Murakozy.pdf}

\includepdf[addtotoc={1, subsubsection, 1, Poster,P16}]{Posters/P16Murakozi.pdf}

\newpage{}
\phantomsection
\stepcounter{articleid}
\addcontentsline{toc}{subsection}{Orosz et~al. Paired plot water balance experiment setup in the botanic garden of the University of Sopron}
\begin{flushleft}

\abstrtitle{Paired plot water balance experiment setup in the botanic garden of the University of Sopron}

\name{Kamilla Orosz$^1$, András Herceg$^1$, Péter Kalicz$^1$, Katalin Anita Zagyvai-Kiss$^1$, Klaudija Lebar$^2$, Katarina Zabret$^2$, Nejc Bezak$^2$, Mark Bryan Alivio$^2$, Gábor Keve$^3$, Dániel Koch$^3$, Zoltán Gribovszki$^1$}

\index{Orosz, Kamilla|textit}
\index{Herceg, András}
\index{Kalicz, Péter}
\index{Zagyvai-Kiss, Katalin}
\index{Lebar, Klaudija}
\index{Zabret, Katarina}
\index{Bezak, Nejc}
\index{Alivio, Mark}
\index{Keve, Gábor}
\index{Koch, Dániel}
\index{Gribovszki, Zoltán}
\institute{$^1$ Institute of Geomatics and Civil Engineering, University of Sopron, Hungary}

\institute{$^2$ Faculty of Civil and Geodetic Engineering, University of Ljubljana, Slovenia}

\institute{$^3$ Faculty of Water Sciences, University of Public Service, Hungary}

\email{OroszK20@uni-sopron.hu}

\end{flushleft}

\noindent
 
Paired plot-based hydrological measurements present good opportunities for water balance comparisons of different surface covers. Hydro-meteorological measurements have been carried out in the Botanical Garden of the University of Sopron since 1925. Conducting long-term measurements presents an excellent opportunity to establish studies on forest water balance. The proximity of educational facilities provides ample opportunities for frequent measurements, routine equipment checks, and student involvement. A paired-plot (clearing, black pine forest") hydrological experiment was set up within the framework of an international Slovenian-Hungarian project. The first set of experiments are suitable for analysing rainfall distribution in a forest (crown and litter interception), but they are also suitable for comparing the soil moisture and groundwater dynamics of grass and forest plots. The process of installing automated equipment in the experimental area is currently in progress. Beyond its primary research character, the experiment also serves educational and demonstration purposes.

The following joint projects (143972SNN, N2-0313 projects and the TKP2021-NKTA-43 project) supported the preparation of this paper. TKP\-2021-NKTA-43 has been implemented with the support provided by the Ministry of Innovation and Technology of Hungary (successor: Ministry of Culture and Innovation of Hungary) from the National Research, Development and Innovation Fund, financed under the TKP2021-NKTA funding scheme. This contribution is also part of ongoing research entitled “Microscale influence on runoff” supported by the Slovenian Research and Innovation Agency (N2-0313).

\includepdf[pages = -, pagecommand={}, addtotoc={1, subsubsection, 1, Short article,A17}]{ShortArt/P17Oroszetal.pdf}

\includepdf[addtotoc={1, subsubsection, 1, Poster,P17}]{Posters/P17Orosz.pdf}

\newpage{}
\phantomsection
\stepcounter{articleid}
\addcontentsline{toc}{subsection}{Herceg et~al. Impact of climate variability on a crown interception of a beech forest}
\begin{flushleft}

\abstrtitle{Impact of climate variability on a crown interception of a beech forest}

\name{András Herceg$^1$, Péter Kalicz$^1$, Nejc Bezak$^2$, Klaudija Lebar$^2$, Katarina Zabret$^2$, Katalin Anita Zagyvai-Kiss$^2$, Zoltán Gribovszki$^1$}

\index{Herceg, András|textit}
\index{Kalicz, Péter}
\index{Bezak, Nejc}
\index{Lebar, Klaudija}
\index{Zabret, Katarina}
\index{Zagyvai-Kiss, Katalin}
\index{Gribovszki, Zoltán}
\institute{$^1$ Institute of Geomatics and Civil Engineering, University of Sopron, Hungary}

\institute{$^2$ Faculty of Civil and Geodetic Engineering, University of Ljubljana, Slovenia}

\email{herceg.andras@uni-sopron.hu}

\end{flushleft}

\noindent

The role of tree canopies is crucial in forest hydrology, as they intercept significant amounts of precipitation, which is evaporated back into the atmosphere during and after rainfall events. This process determines the net intake of forest soil and is an important factor in a site potential. The average amount of interception loss heavily depend on the storage capacity of tree canopies and the rainfall distribution.

In this study a site-based interception estimation based on the Meriam model was developed for two characteristic years at a beech forest plot in the Sopron Hills area (near the Hidegvíz Valley experimental catchment of Hungary). Ground-based observations and MODIS LAI datasets were applied to model the seasonal variations in the canopy storage. 

There was 1.5 times difference between the two years in the context of the precipitation sums. The main finding was that the average yearly interception ratios were significantly less (29\%) in one year, when a lower number of precipitation events occurred compared to the year with a higher number of events (39\%). On the other hand, the absolute values of the interception (mm) were nearly the same.

The 143972SNN, N2-0313 and the TKP2021-NKTA-43 joint projects supported the preparation of this paper. TKP2021-NKTA-43 was implemented with support provided by the Ministry of Innovation and Technology of Hungary (successor: Ministry of Culture and Innovation of Hungary) from the National Research, Development and Innovation Fund, which is financed under the TKP2021-NKTA funding scheme. This contribution is also part of ongoing research entitled “Microscale influence on runoff” supported by the Slovenian Research and Innovation Agency (N2-0313).

Keywords: interception, precipitation distribution, beech, water balance

\includepdf[pages = -, addtotoc={1, subsubsection, 1, Short article,A18}]{ShortArt/P18Herceg.pdf}

\includepdf[addtotoc={1, subsubsection, 1, Poster,P18}]{Posters/P18Herceg.pdf}

\newpage{}
\phantomsection
\stepcounter{articleid}
\addcontentsline{toc}{subsection}{Alivio et~al. Diurnal streamflow patterns in small urban mixed forest catchment in Ljubljana, Slovenia}
\begin{flushleft}

\abstrtitle{Diurnal streamflow patterns in small urban mixed forest catchment in Ljubljana, Slovenia}

\name{Mark Bryan Alivio$^1$, Nejc Bezak$^1$, Mojca Šraj$^1$, Zoltán Gribovszki$^2$, Péter Kalicz$^2$}

\index{Alivio, Mark|textit}
\index{Bezak, Nejc}
\index{Sraj, Mojca@\v{S}raj, Mojca}
\index{Gribovszki, Zoltán}
\index{Kalicz, Péter}
\institute{$^1$ University of Ljubljana, Faculty of Civil and Geodetic Engineering, Ljubljana, Slovenia}

\institute{$^2$ University of Sopron, Institute of Geomatics and Civil Engineering, Sopron, Hungary}

\email{malivio@fgg.uni-lj.si}

\end{flushleft}

\noindent

The present study aims to analyze the diurnal streamflow patterns of a small catchment nourished by an urban mixed forest in the city of Ljubljana, Slovenia, with an outlet joining the city’s drainage system. The diurnal patterns of streamflow remain pertinent in understanding the eco-hydrological processes occurring in an urban forest and its intricate response to environmental drivers, which is essential in harnessing their role as an element of nature-based solutions in urban water management strategies. The water level data in the creek, including the stream water temperature, were measured at 10-min intervals (HOBO Fresh Water Level Data Logger), which were converted to discharge data via a rating curve. The stage–discharge relationship was established by performing sporadic discharge measurements during low- and high-flow events. A visual analysis of the 2-year time series data demonstrated a clear diurnal rhythm in streamflows from May to September, particularly during periods of low flows and no precipitation. This pattern is characterized by a gradual rise and sharp decline, often with noticeable amplitudes. From late spring to summer, streamflow maxima generally occurred in the early morning, and the minima were observed in the afternoon, whereas during the winter, this timing is either reversed or the diurnal pattern is less pronounced. Notably, an inverse synchronization between the streamflow and diurnal temperature (air and stream water) cycles, including solar radiation, was apparent from May to September. The temperature (air/stream) maxima correspond to the streamflow minima, and vice versa, with some lag time. Additionally, the streamflow fluctuations were paralleled by fluctuations of the temperature-dependent streamwater’s viscosity, potentially inducing diurnal variations in the hydraulic conductivity. Thus, we hypothesized that the viscosity effect is a relevant process contributing to the diurnal streamflow pattern in this catchment, even when it is overlain by the stronger influence of evapotranspiration. The interplay of the diurnal fluctuations of the different elements could potentially account for the diurnal patterns in the streamflow.

Keywords: diurnal pattern, streamflow, urban forest, viscosity

Acknowledgment: This contribution is part of ongoing research entitled “Microscale influence on runoff” supported by the Slovenian Research and Innovation Agency (N2-0313) and National Research, Development, and Innovation Office (OTKA project grant number SNN143972).

\includepdf[pages = -, addtotoc={1, subsubsection, 1, Short article,A19}]{ShortArt/P19Alivioetal.pdf}

\newpage{}
\phantomsection
\stepcounter{articleid}
\addcontentsline{toc}{subsection}{Zagyvai-Kiss et~al. Surface water-groundwater interaction analysis in a forested riparian zone}
\begin{flushleft}

\abstrtitle{Surface water-groundwater interaction analysis in a forested riparian zone}

\name{Katalin Anita Zagyvai-Kiss$^1$, András Herceg$^1$, Csenge Nevezi$^1$, Péter Kalicz$^1$, Tamás Bazsó$^1$, Gábor Brolly$^1$, Zoltán Gribovszki$^1 $}

\index{Zagyvai-Kiss, Katalin|textit}
\index{Herceg, András}
\index{Nevezi, Csenge}
\index{Kalicz, Péter}
\index{Bazsó, Tamás}
\index{Brolly, Gábor}
\index{Gribovszki, Zoltán}
\institute{$^1$ University of Sopron, Faculty of Forestry, Institute of Geomatics and Civil Engineering}

\email{zagyvaine.kiss.katalin@uni-sopron.hu}

\end{flushleft}

\noindent

Riparian forests are valuable from a nature conservation viewpoint, but a drying climate can threaten them. The groundwater dynamics of these areas provide essential information for a more accurate quantification of the elements of the water balance, as this surplus form of water is usually a prerequisite for the survival of the communities there. This study investigates the surface water-groundwater interaction in a forested riparian zone.

Our study area is a streamside alder forest ecosystem in an experimental catchment of the Hidegvíz Valley in Hungary. Spatial groundwater level dynamics were assessed using data from seven manually (and, in the case of one well, automatically as well) detected groundwater wells during the period 2017-2022. The associated meteorological parameters were collected on an open-air site nearby.

The estimated groundwater transpiration of the alder forest ecosystem was significant during hot periods without any precipitation. Due to the spatial analyses of the streamside zone’s groundwater level, we found that in dry summer periods, the water table can significantly fall below the streambed.

The 143972SNN and the TKP2021-NKTA-43 joint projects supported the preparation of this paper. TKP2021-NKTA-43 has been implemented with the support provided by the Ministry of Innovation and Technology of Hungary (successor: Ministry of Culture and Innovation of Hungary) from the National Research, Development and Innovation Fund, financed under the TKP2021-NKTA funding scheme.

\includepdf[pages = -, addtotoc={1, subsubsection, 1, Short article,A20}]{ShortArt/P20ZKKA.pdf}

\includepdf[addtotoc={1, subsubsection, 1, Poster,P20}]{Posters/P20ZKKA.pdf}

\newpage{}
\phantomsection
\stepcounter{articleid}
\addcontentsline{toc}{subsection}{Kele et~al. Investigation of the groundwater turnover in a salt steppic oak forest in the lowlands}
\begin{flushleft}

\abstrtitle{Investigation of the groundwater turnover in a salt steppic oak forest in the lowlands}

\name{Zsombor Kele$^1$, Péter Kalicz$^1$, Zoltán Gribovszki$^1$}

\index{Kele, Zsombor|textit}
\index{Kalicz, Péter}
\index{Gribovszki, Zoltán}
\institute{$^1$ University of Sopron, Geomatics and Civil Engineering, Hydrology, Hungary}

\email{kele.zsombor@student.uni-sopron.hu}

\end{flushleft}

\noindent

As groundwater levels continue to sink, it is vital to understand what is happening in the groundwater beneath a forest, how much it is taking up, and how much it is taking away from its environment. The oak forest of Ohat is a relict area along the Tisza, so it is also important from a conservation point of view to explore what is happening there.

In 2021, two groundwater wells were installed in the area, and automatic measuring devices were installed in them to record the daily fluctuations of groundwater. We will use the White method to calculate the groundwater recharge during periods of no precipitation, which will allow us to estimate the groundwater abstraction from the forest. We will also examine how the conditions in the area have changed in the light of long-term meteorological data series for the region.

The multi-year data series will allow us to compare the soil water recharge in different years.

The following joint projects (143972SNN project and the TKP2021-NKTA-43 project) supported the preparation of this paper. TKP2021-NKTA-43 has been implemented with the support provided by the Ministry of Innovation and Technology of Hungary (successor: Ministry of Culture and Innovation of Hungary) from the National Research, Development and Innovation Fund, financed under the TKP2021- NKTA funding scheme.

\includepdf[pages = -, addtotoc={1, subsubsection, 1, Short article,A21}]{ShortArt/P21Zsombor.pdf}

\includepdf[addtotoc={1, subsubsection, 1, Poster,P21}]{Posters/P21Kele.pdf}

\newpage{}
\phantomsection
\stepcounter{articleid}
\addcontentsline{toc}{subsection}{Katona et~al. Drought sensitivity and water-holding capacity of lowland soils under forests in the Carpathian Basin}
\begin{flushleft}

\abstrtitle{Drought sensitivity and water-holding capacity of lowland soils under forests in the Carpathian Basin}

\name{Máté Katona$^1$, Péter Végh$^1$, András Bidló$^1$, Adrienn Horváth$^1$}

\index{Katona, Máté|textit}
\index{Végh, Péter}
\index{Bidló, András}
\index{Horváth, Adrienn}
\institute{$^1$ University of Sopron, Institute of Environmental Protection and Nature Conservation, 9400 Sopron Bajcsy-Zsilinszky St. 4. Hungary}

\email{katona.mate@phd.uni-sopron.hu}

\end{flushleft}

\noindent

The changing climate is bringing more extreme weather and the uneven distribution of rainfall events. These effects are already being observed, and although the average of the many annual precipitation totals has not been changing significantly, the length and frequency of periods of no precipitation and drought have significantly increased. These changes are also being felt by forest stands, and their sensitivity to drought is a crucial factor in their management. During our study, we looked for stands with more extreme site conditions and which are more exposed to the effects of drought. Therefore, we wanted to address the water-holding capacity of lowland soils under forests with more extreme weather conditions and investigate the drought sensitivity of these soils.

We have recently investigated the water-holding capacity of the soils of two stands of the "Tilos" forest in Kunpeszér and two stands of a forest near the Peitsik watercourse in Szentkirály, Hungary. The soils were sampled to a depth of 110 cm using a motorized soil sampler.

In the two areas selected, we recorded albic arenosol (calcaric) and petrocalcic chernozem genetic soil types under the pedunculated oak stands; the climatic characteristics indicated a forest-steppe climate. The texture of the soils was sand and loamy sand. The pH, lime content, organic matter content, texture, and bulk density of the soil fractures were measured at a 10 cm resolution under laboratory conditions, from which the potential available water for plants (PAW) was calculated using pedotransfer functions. In addition, using the climatic data of the area and the results obtained, its sensitivity to drought was investigated using the Thornthwaite-type model.

\includepdf[pages = -, addtotoc={1, subsubsection, 1, Short article,A22}]{ShortArt/P22Katona.pdf}

\newpage{}
\phantomsection
\stepcounter{articleid}
\addcontentsline{toc}{subsection}{Koch et~al. Infiltration and soil moisture measurements in an experimental parcel}
\begin{flushleft}

\abstrtitle{Infiltration and soil moisture measurements in an experimental parcel}

\name{Dániel Koch$^1$, Fruzsina Kata Majer$^1$, Endre Dobos$^2$, Katalin Bene$^3$}

\index{Koch, Dániel|textit}
\index{Majer, Fruzsina}
\index{Dobos, Endre}
\index{Bene, Katalin}
\institute{$^1$ University of Public Service, Faculty of Water Sciences, Baja, Hungary }

\institute{$^2$ University of Miskolc, Institute of Geography and Geoinformatics, Miskolc, Hungary}

\institute{$^3$ Széchenyi István University, Department of Transport Infrastructure and Water Resources Engineering, Győr, Hungary}

\email{koch.daniel@uni-nke.hu}

\end{flushleft}

\noindent

With changing weather conditions, understanding the infiltration process is increasingly important. The Faculty of Water Sciences of the University of Public Service in Hungary created an experimental catchment area in the East Mecsek Hill along the Völgységi stream to measure the elements of the rainfall-runoff process. An infiltration experiment was recently carried out in the irrigation parcel to determine the infiltration curve and better understand the infiltration process. During the infiltration experiment, measurements were combined with multi-level soil moisture probes to determine the water content in different soil layers. Throughout the infiltration measurements, two multi-level soil moisture meters were placed within the experimental plot to collect soil moisture data at 10, 20, 30, 40, 60, and 100 cm depths. The results show that the top soil layer has the highest moisture content. There is a clay layer at a depth of about 30–40 cm a with better water-holding capacity and lower water conductivity. Between 60 and 100 cm, the sandy soil-forming rock still receives a small water recharge when intense rainfall occurs, but this layer is slow to respond to surface inputs. In situ and laboratory samples were taken to determine the soil hydraulic parameters.

The research presented in the article was carried out within the framework of the Széchenyi Plan Plus program with the support of the RRF 2.3.1 21 2022 00008 project. 


\includepdf[pages = -, pagecommand={}, addtotoc={1, subsubsection, 1, Short article,A24}]{ShortArt/P23Kochetal.pdf}

\newpage{}
\phantomsection
\stepcounter{articleid}
\addcontentsline{toc}{subsection}{Parajka et~al. IRISE: Impact of rainfall interception on soil erosion}
\begin{flushleft}

\abstrtitle{IRISE: Impact of rainfall interception on soil erosion}

\name{Juraj Parajka$^1 $, Borbála Széles$^1$, Urša Vilhar$^2$, Nejc Bezak$^3$, Mojca Šraj$^3$}

\index{Parajka, Juraj|textit}
\index{Széles, Borbála}
\index{Vilhar, Urša}
\index{Bezak, Nejc}
\index{Sraj, Mojca@\v{S}raj, Mojca}
\institute{$^1$ Institute of Hydraulic Engineering and Water Resources Management, TU Wien, Vienna, Austria}

\institute{$^2$ Slovenian Forestry Institute, Ljubljana, Slovenia}

\institute{$^3$ University of Ljubljana, Faculty of Civil and Geodetic Engineering, Ljubljana, Slovenia}

\email{parajka@hydro.tuwien.ac.at}

\end{flushleft}

\noindent

Rainfall interception by vegetation is an essential part of the hydrological water cycle. Part of the intercepted rainfall evaporates into the atmosphere, and throughfall and stemflow contribute to surface runoff, affect soil erosion and infiltration processes, and control soil moisture and runoff connectivity patterns. The impact of changing climate and land cover on the rainfall interception, structure, velocity and erosive power is, however, still not well understood. 

This contribution presents the objectives of a bilateral research project between TU Wien and the University of Ljubljana that aims to understand the effect of meteorological and vegetation characteristics on changes in a raindrop microstructure and, therefore, on the erosive power of rainfall. The main research focus is to investigate and compare the mechanisms of the rainfall interception process in different climate conditions in terms of the throughfall microstructure below different vegetation types, i.e., trees and crops on the experimental urban plots in Slovenia and a small agricultural basin in Austria and to evaluate the impact of the changed raindrop microstructure on the erosive power of the rainfall as well as to sediment transport in the streams. 

Acknowledgment: This contribution is part of the ongoing research project entitled “Evaluation of the impact of rainfall interception on soil erosion” supported by the Slovenian Research and Innovation Agency (J2-4489) and it was funded in part by the Austrian Science Fund (FWF) I 6254-N.
\newpage{}
\phantomsection
\stepcounter{articleid}
\addcontentsline{toc}{subsection}{Hohenstein et~al. Investigation of erosion in the Hydrological Open Air Laboratory – The HOAL 2.0 Project}
\begin{flushleft}

\abstrtitle{Investigation of erosion in the Hydrological Open Air Laboratory – The HOAL 2.0 Project}

\name{Leon Hohenstein$^1$, Günter Blöschl$^1$, Peter Strauss$^2$, Elmar Schmaltz$^2$, Jürgen Komma$^1$, Juraj Parajka$^1$, Borbála Széles$^1$, Peter Valent$^1$, Carmen Krammer$^2$, Thomas Brunner$^2$}

\index{Hohenstein, Leon|textit}
\index{Blöschl, Günter}
\index{Strauss, Peter}
\index{Schmaltz, Elmar}
\index{Komma, Jürgen}
\index{Parajka, Juraj}
\index{Széles, Borbála}
\index{Valent, Peter}
\index{Krammer, Carmen}
\index{Brunner, Thomas}
\institute{$^1$ Institute of Hydraulic Engineering and Water Resources Management, Vienna University of Technology, Austria}

\institute{$^2$ Federal Agency of Water Management, Institute for Land and Water Management Research, Austria}

\email{leon.hohenstein@baw.at}

\end{flushleft}

\noindent

Pluvial floods can lead to soil erosion, soil loss and increased sediment loads in streams and can result in increased risks and economic losses. In many regions, the likelihood of such events is growing due to climate change.

The Hydrological Open Air Laboratory (HOAL) in Petzenkirchen in Lower Austria is an experimental catchment that has offered long-term observations of hydro-meteorological variables since 1945 that enable the investigation of the drivers and factors which control erosion processes. Additionally, its geographical and agricultural characteristics are representative of regions all over the world. The HOAL 2.0 project aims to answer the following questions: (1) To what extent can optimal and integrative agricultural land management contribute to the effective water retention of landscapes and therefore reduce the risk of pluvial extreme discharges in agricultural catchments? (2) To which way can an integrative and participatory implementation of optimized land-management practices be realized for achieving maximal protection as well as maximal acceptance by farmers, mayors, residents and other stakeholders?

In a first step, the potential of different on and off-site land-management practices are evaluated. The hydro-meteorological monitoring network in HOAL captures all the components of the water cycle in a high spatio-temporal resolution, which enables answering the research issues of the project. The measurements can then provide a solid basis for developing models for the generalization of the results. Finally, a comprehensive communication and outreach campaign can allow for a public exchange about the findings and encourage further discussion with farmers, stakeholders and residents. This will ensure that the best practices for land management are realistic and economically feasible.
\newpage{}
\phantomsection
\stepcounter{articleid}
\addcontentsline{toc}{subsection}{Grečnárová and Danáčová. The experimental measurements of the effectiveness of roof substrate in terms of a reduction in runoff from vegetated roofs}
\begin{flushleft}

\abstrtitle{The experimental measurements of the effectiveness of roof substrate in terms of a reduction in runoff from vegetated roofs}

\name{Jana Grečnárová$^1$, Michaela Danáčová$^1$}

\index{Grečnárová, Jana|textit}
\index{Danáčová, Michaela}
\institute{$^1$ Department of Land and Water Resources Management, Slovak University of Technology, Faculty of Civil Engineering, Radlinského 11, 810 05, Bratislava, Slovakia}

\email{jana.grecnarova@stuba.sk}

\end{flushleft}

\noindent

The primary focus of the experiment is to investigate the substrate material by examining its composition and properties. Its main objective is to investigate and compare how different intensities of short-term extreme rainfall events affect the ability of the substrate to retain water under different conditions. 

Specifically, the efficiency of a pure substrate is compared with substrates enriched with 5\% and 10\% of the biochar components. The results achieved from this experiment reveal differences in the adsorption capacity of the three different samples. The addition of biochar shows an especially significant effect on the delay of runoff from the substrate, which was mainly observed during a 15-minute of low-intensity rainfall. The most significant differences in water retention were observed during a 15-minute rainfall with an intensity of 3 mm/min. In contrast, the differences in retention appear less significant at more moderate rainfall intensities (4 mm/min) and even at extreme rainfall intensities (5 mm/min). 

Based on the observations from the experimental measurements, the use of biochar as an alternative method to improve the retention capacity of a substrate proves to be a reasonable option for reducing rainfall runoff.

Keywords: roof substrate, biochar, runoff, water retention, rainfall intensity 

Acknowledgements:
The study was supported by the VEGA grant project No. 1/4021/07.

\includepdf[addtotoc={1, subsubsection, 1, Poster,P26}]{Posters/P26.pdf}

\newpage{}
\phantomsection
\stepcounter{articleid}
\addcontentsline{toc}{subsection}{Zaťovičová and Majorošová. A comparison of the Surface temperature of typical urban materials in relation to the street orientation}
\begin{flushleft}

\abstrtitle{A comparison of the Surface temperature of typical urban materials in relation to the street orientation}

\name{Miriam Zaťovičová$^1$, Martina Majorošová$^1$}

\index{Zaťovičová, Miriam|textit}
\index{Majorošová, Martina}
\institute{$^1$ Slovak University of Technology in Bratislava, Department of Land and Water Resource Management}

\email{miriam.zatovicova@stuba.sk}

\end{flushleft}

\noindent

The overheating of cities is mainly caused by sealing formerly permeable surfaces and eliminating vegetation. Even though greenery is a valuable asset for mitigating extreme heat conditions and has proved to be the most efficient, it is vital to comprehend the urban conditions to create effective strategies with positive microclimatic and economic merit. The inappropriate use of trees can result in unnecessary expenses, but more importantly, it can worsen urban climate conditions by obstructing the airflow and subsequent entrapment of the overheated air. Therefore, it is essential to examine the local conditions and determine the most irradiated and overheated parts of streets suitable for tree planting. Subsequently, combining these measures with the existing building geometry and using building shading could be a suitable tool for overheating mitigation. When applied correctly, this combination has proved to be the most effective by creating shade and enabling airflow. In our pilot case study, we measured the surface temperature of the five most typical materials in Old Town district of Bratislava, Slovakia, in relation to their cardinal directions. We chose the representative streets in each direction and divided them into two groups i.e., narrow and wide. Consequently, we compared the temperature progression of each material in relation to their orientation, geometry, and the time of day. The results could serve as a basis for our further studies of urban conditions and provide a general key to effective blue-green infrastructure strategies.

This study has been supported by the Scientific Grant Agency under Contracts no. VEGA 1/0067/23.

\includepdf[pages=1-6,pagecommand={}, nup=1x2, width=\textwidth, delta=0 20mm, addtotoc={1, subsubsection, 1, Presentation,Pe27}]{Presis/P27Zatovicova.pdf}

\includepdf[addtotoc={1, subsubsection, 1, Poster,P27}]{Posters/P27.pdf}

\newpage{}
\phantomsection
\stepcounter{articleid}
\addcontentsline{toc}{subsection}{Kálmán and Bene. The potential economic benefits of rainwater in the ``green city development'' program}
\begin{flushleft}

\abstrtitle{The potential economic benefits of rainwater in the ``green city development'' program}

\name{Attila Kálmán$^1$, Katalin Bene$^1$}

\index{Kalman, Attila@K\'alm\'an, Attila|textit}
\index{Bene, Katalin}
\institute{$^1$ National Laboratory for Water Science and Water Security, Széchenyi István University, Department of Transport Infrastructure and Water Resources Engineering, Egyetem square 1., H-9026 Győr, Hungary}

\email{kalman.attila@sze.hu}

\end{flushleft}

\noindent

The adverse effects of climate change impact our daily lives and challenge decision-makers to address it. The primary focus of recent decades was mitigation, but recently, researchers and stakeholders have prioritized adaptation measures instead of mitigation efforts. To achieve effective adaptation measures, it is important to economically evaluate natural resources and possible scenarios with or without interventions. A comparison of potential damage and lost profits with achievable benefits is necessary for effective decision-making. Governments and municipalities have started to recognize the importance of adaptation actions, which have boosted various ``green city, green settlement'' type funded programs.

Hundreds of blue-green infrastructure developments have been completed in Hungary's most extensive regional development program, known as ``green city development''. This study focuses on the potential rainwater retention and the economically achievable benefits of these infrastructures. A part of the Central Transdanubian region was selected from these developments for evaluation. The maximum retainable rainwater and the theoretically attainable economic benefits were determined. The best methods were collected based on these criteria, and applicable nature-based solutions were proposed for using the stored (retained) water. 

keywords: blue-green infrastructure, nature-based solutions, water retention, climate adaptation, water value, economic benefits

\includepdf[addtotoc={1, subsubsection, 1, Poster,P28}]{Posters/P28.pdf}

\newpage{}
\phantomsection
\stepcounter{articleid}
\addcontentsline{toc}{subsection}{Clement et~al. Implementation of a monitoring program to track pollutant transport processes in a pilot catchment in Hungary}
\begin{flushleft}

\abstrtitle{Implementation of a monitoring program to track pollutant transport processes in a pilot catchment in Hungary}

\name{Adrienne Clement$^1$, József Deák$^2$, Bence Decsi$^1$, István Gábor Hatvani$^3$, Zsolt Jolánkai$^1$, Zoltán Kern$^3$, Máté Krisztián Kardos$^1$, Zsolt Kozma$^1$, László Palcsu$^4$, Sonja Lojen$^5$, Radmila Milačič$^5$, Polona Vreča$^5$}

\index{Clement, Adrienne|textit}
\index{Deák, József}
\index{Decsi, Bence}
\index{Hatvani, István}
\index{Jolánkai, Zsolt}
\index{Kern, Zoltán}
\index{Kardos, Máté}
\index{Kozma, Zsolt}
\index{Palcsu, László}
\index{Lojen, Sonja}
\index{Milačič, Radmila}
\index{Vreča, Polona}
\institute{$^1$ Budapest University of Technology and Economics, Hungary}

\institute{$^2$ GWIS Ltd., Hungary}

\institute{$^3$ HUN-REN Research Centre for Astronomy and Earth Sciences, Hungary}

\institute{$^4$ HUN-REN Institute for Nuclear Research, Hungary}

\institute{$^5$ Jožef Stefan Institute, Slovakia}

\email{adrienne.clement@emk.bme.hu}

\end{flushleft}

\noindent

To validate catchment-scale emission models, a targeted measurement program was implemented in the Koppány catchment (660 km$^2$). Samples were collected from stream water (the Koppány and its tributaries and shallow groundwater along the riverbed. Monthly bulk precipitation was collected. In addition to the water chemistry, stable and radioactive isotopes were measured as natural tracers to investigate hydrological pathways and residence times. Seasonal variations of water stable isotopes in the precipitation and stream water were observed to estimate the stream residence time. Tritium measurements were performed in the shallow groundwater zone to estimate the age of the groundwater. Preliminary results suggest that although the dominant land use in the watershed is arable land, during dry periods, the effect of the wastewater load determines the water quality of the Koppány-creek. Agricultural nitrate pollution is conveyed by the tributaries, and the nitrogen load contribution of the groundwater is negligible. The primary source of the particle-bound pollutants is the soil washed away during high-flow events. With the study, excess information can be obtained for a better understanding of pollutant transport via surface runoff and subsurface pathways.

The research is supported by the National Research Development and Innovation Office (NKFIH) through the OTKA Grant SNN 143868.

\includepdf[addtotoc={1, subsubsection, 1, Poster,P29}]{Posters/P29.pdf}

\newpage{}
\phantomsection
\stepcounter{articleid}
\addcontentsline{toc}{subsection}{Kerék et~al. Riverbed incision on the upper Hungarian Danube and Raab rivers in the Moson-Danube confluence area}
\begin{flushleft}

\abstrtitle{Riverbed incision on the upper Hungarian Danube and Raab rivers in the Moson-Danube confluence area}

\name{Gábor Kerék$^1$, Gábor Keve$^2$, Dorottya Szám$^3$, Enikő Anna Tamás$^4$}

\index{Kerék, Gábor|textit}
\index{Keve, Gábor}
\index{Szám, Dorottya}
\index{Tamás, Enikő}
\institute{$^1$ Hydrologist, North-Transdanubian Water Directorate, 9021 Győr, Hungary, e-mail: kerek.gabor@eduvizig.hu}

\institute{$^2$ Associate professor and head of department, Department of Regional Water Management, Ludovika University of Public Service, Faculty of Water Sciences, 6500 Baja, Hungary, e-mail: keve.gabor@uni-nke.hu}

\institute{$^3$ Chief Project Officer, Department of Regional Water Management, Ludovika University of Public Service, Faculty of Water Sciences, 6500 Baja, Hungary, e-mail: szam.dorottya@uni-nke.hu}

\institute{$^4$ Professor, Department of Regional Water Management, Ludovika University of Public Service, Faculty of Water Sciences, 6500 Baja, Hungary, e-mail: tamas.eniko.anna@uni-nke.hu}

\email{szam.dorottya@uni-nke.hu}

\end{flushleft}

\noindent

The phenomenon of river bed subsidence on our channelized rivers has been a well-known fact for a long time. In the upper reaches of the River Danube in Hungary this process seems to have accelerated in recent decades due to the construction of hydroelectric power plants on the German, Austrian, and Slovakian sections of the river and other anthropogenic effects. As a result, similar processes have also occurred on the lower reaches of the Moson-Danube and Raab rivers, which are significantly hydromorphologically influenced by the Danube. In our research, we measured the magnitude and temporal changes of this phenomenon by examining the hydrological time series for the Danube between the gauging stations of Dévény and Komárom and that of the Raab between Sárvár and Győr. 

We pointed out that the phenomenon of bed erosion is still measurable in the upper reaches of the Danube in Hungary. In 2022, the Moson-Danube dam was inaugurated to restore low and medium water levels in the Moson-Danube and Rába estuaries, rehabilitate wetlands, ensure navigation, and enhance flood protection. After the first year of the dam’s operation, there was a visible increase in the water level at the confluence of the Moson-Danube and Raab rivers, which solved the primary problem but generated additional known problems, such as handling the settling surplus of suspended sediment from the River Raab and difficulties with sewage disposal in the city of Győr. The dam is evidently unable to deal with issues in the ongoing process of the Danube riverbed incision. The settling of suspended sediment could reduce the magnitude of the erosion on the lower part of the Raab and Moson-Danube rivers but may have harmful side effects for river navigation over the long term. Furthermore, the floodgate function of the estuary structure in the event of a major Danube flood eliminates any backwater effect on the Moson-Danube, which could have a negative impact on the culminating discharges of the Danube on the Gönyű-Komárom stretch. The previous effect of the reservoir on the lower reaches of the Moson-Danube and the Raab will be eliminated by the operation of the Moson-Danube estuary dam; thus higher peak discharges will occur on the Danube below the mouth of the Moson-Danube. This higher discharge could establish higher water velocities and water levels, which may speed up the erosion of the riverbed. This phenomenon should be evaluated during the next flood event as it occurs.

Keywords: hydrology, hydromorphology, trend analysis, flood protection, navigation, river basin, Danube, Raab, riverbed incision

\includepdf[addtotoc={1, subsubsection, 1, Poster,P30}]{Posters/P30.jpg}

\newpage{}
