\phantomsection
\stepcounter{articleid}
\addcontentsline{toc}{subsection}{Zabret et~al. Comparison of rainfall characteristics between two experimental plots in Slovenia and Austria}
\begin{flushleft}

\abstrtitle{Comparison of rainfall characteristics between two experimental plots in Slovenia and Austria}

\name{Katarina Zabret$^1$, Borbala Szeles$^2$, Juraj Parajka$^2$, Mojca Šraj$^1$, Dušan Marjanović$^2$, Nejc Bezak$^1$, Urša Vilhar$^3$, Peter Strauss$^4$, Günter Blöschl$^1$}

\index{Zabret, Katarina|textit}
\index{Szeles, Borbala}
\index{Parajka, Juraj}
\index{Sraj, Mojca@\v{S}raj, Mojca}
\index{Marjanović, Dušan}
\index{Bezak, Nejc}
\index{Vilhar, Urša}
\index{Strauss, Peter}
\index{Blöschl, Günter}
\institute{$^1$ Chair of Hydrology and Hydraulic Engineering, Faculty of Civil and Geodetic Engineering, University of Ljubljana, Slovenia}

\institute{$^2$ Institute of Hydraulic Engineering and Water Resources Management, Vienna University of Technology, Austria}

\institute{$^3$ Slovenian Forestry Institute, Slovenia}

\institute{$^4$ Federal Agency of Water Management, Institute for Land and Water Management Research, Austria}

\email{Corresponding author: szeles@hydro.tuwien.ac.at}

\end{flushleft}

\noindent

The aim of this study was to comparatively assess the rainfall characteristics and erosive potential of rainfall between two locations in the Danube River Basin, i.e., the Austrian Hydrological Open Air Laboratory (HOAL) agricultural catchment and an experimental plot located in an urban park in Ljubljana, Slovenia. The variability of rainfall characteristics and rainfall erosivity were investigated using 5-year-long measurements (2014 – 2018) of precipitation (amount, duration and intensity of rainfall) and drop size distributions (diameter, velocity and median volume diameter of the drops). Despite having the same Köppen-Geiger climate classification, not only similarities but also large differences were found between the two study sites. The long-term annual average total precipitation was almost twice as much in Ljubljana compared to the HOAL. According to the results of the clustering analysis, larger and more intense rainfall events occurred in Ljubljana than in the HOAL, but the average drop characteristics were lower for the events in Slovenia. Additionally, if the events were not extreme, their characteristics were similar regardless of location. The rainfall intensities tended to peak in the summer months at both sites, when rainfall durations were shorter, and larger and faster drops were observed. Due to larger rainfall amounts, the rainfall erosivity was found to be between 2-5 times greater in each year in Ljubljana than in the HOAL. 

Acknowledgment: This contribution is part of the ongoing research project entitled “Evaluation of the impact of rainfall interception on soil erosion” supported by the Slovenian Research and Innovation Agency (J2-4489) and the Austrian Science Fund (FWF) I 6254-N.

Keywords: rainfall, drop size distribution, erosion
\newpage{}
\phantomsection
\stepcounter{articleid}
\addcontentsline{toc}{subsection}{Szám et~al. Analysis of precipitation data series in the TIVIZIG operational area (Hungary)}
\begin{flushleft}

\abstrtitle{Analysis of precipitation data series in the TIVIZIG operational area (Hungary)}

\name{Dorottya Szám$^1$, Zsolt Hetesi$^2$, Tibor Bódi$^3$, Zoárd Ivor Marosi$^4$, Gábor Keve$^5$, Dániel Koch$^6$}

\index{Szam, Dorottya@Sz\'{a}m, Dorottya|textit}
\index{Hetesi, Zsolt}
\index{Bodi, Tibor@B\'{o}di, Tibor}
\index{Marosi, Zoárd}
\index{Keve, Gábor}
\index{Koch, Dániel}
\institute{$^1$ Chief Project Officer, Department of Regional Water Management, Ludovika University of Public Service, Faculty of Water Sciences, 6500 Baja, Hungary, e-mail: szam.dorottya@uni-nke.hu}

\institute{$^2$ Associate professor, Department of Water and Environmental Security, Ludovika University of Public Service, Faculty of Water Sciences, 6500 Baja, Hungary, e-mail: zsolt.hetesi@uni-nke.hu}

\institute{$^3$ Department of Water and Environmental Security, Ludovika University of Public Service, Faculty of Water Sciences, 6500 Baja, Hungary, e-mail: bodi.tibor@stud.uni-nke.hu}

\institute{$^4$ Acting director, Trans Tisza Water Directorate, Debrecen, Hungary, e-mail: titkarsag@tivizig.hu}

\institute{$^5$ Associate professor, Department of Regional Water Management, Ludovika University of Public Service, Faculty of Water Sciences, 6500 Baja, Hungary, e-mail: keve.gabor@uni-nke.hu}

\institute{$^6$ Assistant lecturer, Department of Regional Water Management, Ludovika University of Public Service, Faculty of Water Sciences, 6500 Baja, Hungary, e-mail: koch.daniel@uni-nke.hu}

\email{Corresponding author: szam.dorottya@uni-nke.hu}

\end{flushleft}

\noindent

The Trans-Tisza Regional Water Directorate (TIVIZIG) area is one of the regions of the Carpathian Basin most exposed to the adverse effects of droughts on agriculture. Climate change here increases the likelihood of extreme weather events. It is also true that the frequency of drought and extreme precipitation periods is rising. A better understanding of all of these factors is essential for water planning and control of water management to create the conditions of adaptive and precise farming. This paper's research used daily rainfall data from 16 Trans-Tisza Regional Water Directorate hydro-meteorological stations between 1964 and 2022. We have shown that the change in extreme precipitation events is not statistically significant for the aggregated data of the 16 stations. However, significant changes were detected after splitting the data series into two periods of equal length (1965-1993; 1994-2022) for further comparative statistical analysis. We showed a significant increase in the frequency of periods of at least 25 days without precipitation, according to the data of the hydro-meteorological station in the Nagybajom municipality. These findings, combined with the area's soil conditions, suggests that precise irrigation management is becoming essential to achieve desirable agricultural yields and that, where appropriate, it may be necessary to drain low-lying areas at risk from inland flooding.

Keywords: precipitation, Trans-Tisza region, extreme weather events, drought, climate change
\newpage{}
\phantomsection
\stepcounter{articleid}
\addcontentsline{toc}{subsection}{Sleziak et~al. Accuracy of viirs snow product for mapping of regional and catchment scale snow cover}
\begin{flushleft}

\abstrtitle{Accuracy of viirs snow product for mapping of regional and catchment scale snow cover}

\name{Patrik Sleziak$^1$, Michal Danko$^1$, Martin Jančo$^1$, Ladislav Holko$^1$, Juraj Parajka$^2$}

\index{Sleziak, Patrik|textit}
\index{Danko, Michal}
\index{Jančo, Martin}
\index{Holko, Ladislav}
\index{Parajka, Juraj}
\institute{$^1$ Institute of Hydrology, Slovak Academy of Sciences, Bratislava, Slovakia}

\institute{$^2$ Institute of Hydraulic Engineering and Water Resources Management, TU Wien, Vienna 1040, Austria}

\email{Corresponding author: sleziak@uh.savba.sk}

\end{flushleft}

\noindent

The Moderate (MODIS) Resolution Imaging Spectroradiometer is one of the most attractive remote
sensing datasets used for mapping snow cover. The MODIS product is expected to be replaced by
the Visible Infrared Imaging Radiometer Suite (VIIRS) snow cover product in the near future.
Therefore, a reliably accurate evaluation of this product is needed for future hydrological
applications.

The main objective of this study was to evaluate the accuracy of the VIIRS snow product over the
whole territory of Austria as well as in a small experimental catchment (the Jalovecký Creek
catchment, northern Slovakia) by using extensive snow course measurements at open and forested
sites during the 2012–2022 period. Snow cover is identified by the VIIRS product using the

Normalized Difference Snow Index (NDSI) instead of the simple binary information provided about
snow cover by the former products. We investigated which NDSI threshold should be used for the
VIIRS snow cover data based on ground level snow depth observations. The preliminary results
indicate that the NDSI thresholds decrease with increasing elevations and are lower in forested than
in open land cover types. Our contribution will discuss the factors that control of the mapping
accuracy.

Acknowledgments
This work was supported by the Slovak Research and Development Agency under the Contract no.
APVV-23-0332 and the VEGA Grant Agency no. 2/0019/23.

Keywords: VIIRS, snow, NDSI threshold
\newpage{}
\phantomsection
\stepcounter{articleid}
\addcontentsline{toc}{subsection}{Adam et~al. Effects of Climate and Rainfall Spatio-Temporal Variations on Sudan Ecosystems and Plant Distribution}
\begin{flushleft}

\abstrtitle{Effects of Climate and Rainfall Spatio-Temporal Variations on Sudan Ecosystems and Plant Distribution}

\name{Abdelbagi Yanes F. Adam$^{1,2}$, Mohamed B.O. Osman$^{1,2}$, Zoltán Gribovszki$^1$, Péter Kalicz$^1$}

\index{Adam, Abdelbagi|textit}
\index{Osman, Mohamed}
\index{Gribovszki, Zoltán}
\index{Kalicz, Péter}
\institute{$^1$ Institute of Geomatics and Civil Engineering, University of Sopron, 9400 Sopron, Hungary}

\email{Corresponding author: abdoyanes2016@gmail.com}

\end{flushleft}

\noindent

To better understand the earth, scientists have historically developed standards for dividing its surface into homogeneous physical and biological zones. Consequently, the Earth's surface is classified into well-known ecological and climatic zones. Rainfall and temperature are the key factors that define the characteristics and boundaries of these ecological zones. Since ecology and climates are interconnected, they influence each other. This study highlights the shifts observed in precipitation patterns, which are characterized by erratic rainfall and prolonged drought periods that disrupt plant ecosystems. This paper aims to discover and project potential changes in ecological zones due to shifting climatic conditions in Sudan. It reviews the impacts of climate variability on precipitation distribution and plant ecosystems by focusing on changes in both rainfall and temperature. According to the findings, Sudan's ecological systems have changed. Desert areas have increased by 11\%; semi-desert areas have decreased by 8\%; the area of poor savannas has increased by 21\%, and the area of rich savannas has decreased by 23\% from 1958 to 2010. There was also a decrease in rainfall, with high temporal and spatial variability, and an increase in the temperature. This has significant implications for the distribution and abundance of plant species, potentially leading to shifts in vegetation communities and ecosystem services.

Acknowledgement This article was made within the frame of Project No 143972SNN (OTKA).

Keywords: Climate Trends, Variability, Ecological zone, Sudan, GPM, Geospatial Techniques.
\newpage{}
\phantomsection
\stepcounter{articleid}
\addcontentsline{toc}{subsection}{Muraközy et~al. Water balance and precipitation analysis of long-term data from a meteorological station in Sopron, Hungary}
\begin{flushleft}

\abstrtitle{Water balance and precipitation analysis of long-term data from a meteorological station in Sopron, Hungary}

\name{Lili Muraközy$^{1,2}$, Péter Kalicz$^1$, Zoltán Gribovszki$^1$, Kamila Hlavčová$^2$, Jan Szolgay$^2$}

\index{Muraközy, Lili|textit}
\index{Kalicz, Péter}
\index{Gribovszki, Zoltán}
\index{Hlavčová, Kamila}
\index{Szolgay, Jan}
\institute{$^1$ University of Sopron, Faculty of Forestry, Institute of Geomatics and Civil Engineering}

\institute{$^2$ Slovak University of Technology in Bratislava, Faculty of Civil Engineering, Department of Land and Water Resources Management}

\email{Corresponding author: Lili Muraközy, lillancs01@gmail.com}

\end{flushleft}

\noindent

Long-term meteorological data are crucial for understanding the hydrological impacts of climate change on forests. During our work, data from the Sopron Botanical Garden Climate Station were digitized and error-corrected, and various forest and hydrometeorological studies were carried out on the unified database. The main focus was to observe the changes in precipitation (rain and snow) and the water balance over time, where we observed an increase in the amount of 20+ mm precipitation, but a decrease in the proportion of light rainfall, the most drastic decrease in the runoff potential of the water balance elements (58.7\%), and a decrease in snow depths and storage, as evidenced by the model and the data measured.

Acknowledgements: This study was financially supported by the Slovak Research and Development Agency under Contract No. APVV 23-0332.

Keywords: hydrology, forest, long-time measurements

\newpage{}
\phantomsection
\stepcounter{articleid}
\addcontentsline{toc}{subsection}{Fitriana et~al. Effects of Low Flow, Droughts and Warming on the Ecological State and Functioning of River Systems}
\begin{flushleft}

\abstrtitle{Effects of Low Flow, Droughts and Warming on the Ecological State and Functioning of River Systems}

\name{Farhana Sweeta Fitriana$^1$, Gregor Laaha$^1$, Gabriele Weigelhofer$^2$, Svenja Fischera$^3$}

\index{Fitriana, Farhana|textit}
\index{Laaha, Gregor}
\index{Weigelhofer, Gabriele}
\index{Fischera, Svenja}
\institute{$^1$ BOKU University, Institute of Statistics}

\institute{$^2$ BOKU University, Institute of Hydrobiology and Water Management}

\institute{$^3$ Wageningen University \& Research, Hydrology and Environmental Hydraulics Group}

\email{Corresponding author: farhana.fitriana@boku.ac.at}

\end{flushleft}

\noindent

Low-flow and drought pose significant challenges for water management, impacting water availability and quality. However, the responses of river systems to droughts under climate change remain poorly understood. My PhD aims to investigate the vulnerability and resilience of river systems as a basis for low-flow and drought management plans. The project focuses on Eastern Austria and uses data from 175 gauged catchments with additional local data for an in-depth catchment study. It will be conducted in three stages: (i) characterising temporal dynamics, average, and extreme conditions of streamflow and temperature in river networks; (ii) monitoring of water quality indices and ecological indicators, and (iii) investigation of their interrelationships. I present an ongoing study aiming to extend regional low flow frequency analysis to nonstationary conditions for a better description of extreme events. I first analyse temporal trends in the study area and investigate whether they can be related to temperature increases or other climate trends. I then discuss modelling concepts to extend frequency analysis to nonstationary conditions. The study will serve the overall aim of providing a new perspective on drought processes and impact chains in river systems.

Keywords: low flow, drought, climate change, water quality, non-stationary frequency analysis.

\newpage{}
\phantomsection
\stepcounter{articleid}
\addcontentsline{toc}{subsection}{Štefunková. Evaluating climate change scenarios for hydrological forecasting in Slovakia}
\begin{flushleft}

\abstrtitle{Evaluating climate change scenarios for hydrological forecasting in Slovakia}

\name{Zuzana Štefunková$^1$}

\index{Stefunková, Zuzana@\v{S}tefunková, Zuzana|textit}
\institute{$^1$ Slovak University of Technology, Department of Land and Water Resources Management }

\email{Corresponding author: zuzana\_stefunkova@stuba.sk}

\end{flushleft}

\noindent

Climate scenarios are essential for predicting future changes in temperatures and precipitation, which significantly impact hydrological processes and water resources. This study evaluates climate change trends in Slovakia using data from climate and precipitation stations across the country. The original measurements, which are available up to 2010, served as input for projections extending to 2100. These scenarios have been updated based on new measurements, thereby providing insights into the accuracy of climate predictions.

Building on previous studies that emphasized the importance of historical data, a comparative analysis of updated scenarios from the KNMI and MPI models was conducted for selected stations in Slovakia. The primary goal is to assess the precision of these scenarios and their applicability for hydrological forecasting amid changing climatic conditions.

The results highlight the need for regular monitoring and updating of climate data to ensure reliable predictions. Additionally, this study underscores the importance of effective water resource management in response to climate shifts. By enhancing our understanding of local climates, we can better prepare for climate change challenges and implement adaptive strategies for sustainable water resources in Slovakia. 

Acknowledgements: This study was financially supported by the Slovak Research and Development Agency under Contract No. VEGA 1-0067/23

Keywords: Climate Change, Hydrological Forecasting, Water Resources Management
\newpage{}
\phantomsection
\stepcounter{articleid}
\addcontentsline{toc}{subsection}{Valent et~al. Assessing water balance changes due to climate change with distributed hydrological modelling in Austria}
\begin{flushleft}

\abstrtitle{Assessing water balance changes due to climate change with distributed hydrological modelling in Austria}

\name{Peter Valent$^1$, Juraj Parajka$^1$, Jürgen Komma$^1$}

\index{Valent, Peter|textit}
\index{Parajka, Juraj}
\index{Komma, Jürgen}
\institute{$^1$ Vienna University of Technology, Institute of Hydraulic and Water Resources Engineering}

\email{Peter Valent, valent@hydro.tuwien.ac.at}

\end{flushleft}

\noindent

Austria has set ambitious goals for its expansion of renewable energy and climate protection. By 2030, the country aims for renewables to cover 100\% of its electricity demand, with a target of full climate neutrality by 2040. To support this transition, a distributed rainfall-runoff model was developed to evaluate the technical and economic potential of Austrian surface water bodies under current (1990–2020) and future climate conditions (2040 and 2070) in more than 10,000 river profiles. The model operates on a daily time step and was calibrated to accurately represent monthly water balances. Future climate conditions were assessed using MPI-M-MPI-ESM-LR\_r1i1p1\_SMHI-RCA4 under the RCP 4.5 and RCP 8.5 scenarios. A delta-change approach was applied to adjust historical air temperature and rainfall records, thereby minimizing biases associated with climate model projections. The model performed well on a wide range of the catchments and regions identified for which significant changes in the water balance is expected.

Keywords: distributed rainfall-runoff modelling, water balance, climate change assessment, delta change, ÖKS15, EURO-CORDEX
\newpage{}
\phantomsection
\stepcounter{articleid}
\addcontentsline{toc}{subsection}{Tanhapour et~al. Assessment of streamflow variability through multi-objective calibration approach and climate change}
\begin{flushleft}

\abstrtitle{Assessment of streamflow variability through multi-objective calibration approach and climate change}

\name{Mitra Tanhapour$^1$, Kamila Hlavčová$^1$, Jan Szolgay$^1$}

\index{Tanhapour, Mitra|textit}
\index{Hlavčová, Kamila}
\index{Szolgay, Jan}
\institute{$^1$ Department of Land and Water Resources Management, Faculty of Civil Engineering, Slovak University of Technology, Bratislava, Slovakia}

\email{Corresponding author: mitra.tanhapour@stuba.sk}

\end{flushleft}

\noindent

This project establishes a framework to explore the impact of remote
sensing data and climate models on streamflow variability. In this
regard, historical meteorological data, regional climate models
(RCMs), and remote sensing datasets from MODIS snow cover and ASCAT
soil moisture data are organized for the Hron catchment. These data
sources support a novel rainfall-runoff model calibration process,
where remote sensing data enhance model parameterization by accurately
capturing seasonal snow cover and soil moisture variability. A
multi-objective calibration strategy is applied to balance streamflow
accuracy with satellite-derived snow cover and soil moisture dynamics,
allowing the model to better simulate diverse hydrological
processes. Additionally, an ensemble simulation framework is developed
to project streamflow under different climate scenarios to represent
shifts in hydrological behavior and patterns. The results offer
insights into the contribution of remote sensing data on improving
streamflow simulation and the potential impacts of climate change on
hydrological regimes.

Acknowledgments: This work was supported by the VEGA Grant Agency No. VEGA
1/0577/23 and No. VEGA 1/0782/21.

Keywords: streamflow projection, climate models, soil moisture, multi-objective calibration
\newpage{}
\phantomsection
\stepcounter{articleid}
\addcontentsline{toc}{subsection}{Leščešen et~al. Monthly River Discharge Predictions Using LSTM: A Comparative Study of the Velika Morava River (Serbia) and Morava River (Slovakia)}
\begin{flushleft}

\abstrtitle{Monthly River Discharge Predictions Using LSTM: A Comparative Study of the Velika Morava River (Serbia) and Morava River (Slovakia)}

\name{Igor Leščešen$^1$, Pavla Pekárová$^1$, Pavol Miklánek$^1$, Zbyňek Bajtek$^1$, Veronika Bačová Mitková$^1$}

\index{Leščešen, Igor|textit}
\index{Pekárová, Pavla}
\index{Miklánek, Pavol}
\index{Bajtek, Zbyňek}
\index{Mitková, Veronika}
\institute{$^1$ Institute of Hydrology, Slovak Academy of Sciences, Dúbravská cesta 9, 841 04 Bratislava, Slovakia }

\email{Coresponding author: igorlescesen@yahoo.com}

\end{flushleft}

\noindent

In this study, a Long Short-Term Memory (LSTM) neural network is used to predict streamflow in two important tributaries of the Danube, i.e., the Morava in Slovakia and the Velika Morava in Serbia. Using 60 years of monthly hydrological and climatic data (1961–2020), we investigated different lag times to optimize the model performance. The results show that the optimal delay for the Morava River is 1 month and yields a mean absolute error (MAE) of 37.3, a root mean square error (RMSE) of 55.8, and a coefficient of determination (R²) of 0.4. In contrast, the Velika Morava River has a 6-month delay, with an MAE of 70.5, an RMSE of 107.5 and an R² of 0.6 in the validation. These results highlight the significant differences in the hydrological response between the two rivers, which are influenced by geographical and climatic factors. Understanding these dynamics is crucial for improving river forecasting, which is essential for effective water resource management and mitigating the impact of hydrological extremes such as floods and droughts in the respective regions.

Acknowledgment: This research was supported by Project 09I03-03-V04-00186 "Streamflow Drought Through Time," funded by the European Union's Next Generation EU as part of the Recovery and Resilience Plan of the Slovak Republic, specifically through the initiative for more efficient management and enhancement of the funding for research, development, and innovation.

Keywords: LSTM Neural Network, River Flow Forecasting, Hydrological Extremes, Lag Time Analysis, Danube Tributaries
\newpage{}
\phantomsection
\stepcounter{articleid}
\addcontentsline{toc}{subsection}{Papp et~al. Flood Measurements and Runoff Analysis on the Magyaregregy Experimental Catchment}
\begin{flushleft}

\abstrtitle{Flood Measurements and Runoff Analysis on the Magyaregregy Experimental Catchment}

\name{Bence Papp$^1$, Dániel Koch$^{1,2}$, Enikő Anna Tamás$^{1,2}$, Gábor Keve$^{1,2}$, Dorottya Szám$^{1,2}$}

\index{Papp, Bence|textit}
\index{Koch, Dániel}
\index{Tamás, Enikő}
\index{Keve, Gábor}
\index{Szam, Dorottya@Sz\'{a}m, Dorottya}
\institute{$^1$ Ludovika University of Public Service, Hungary, Faculty of Water Sciences, Baja, Hungary,}

\institute{$^2$ National Laboratory for Water Science and Water Security, Hungary}

\email{Corresponding author: bence.pappisk@gmail.com}

\end{flushleft}

\noindent

The aim of our poster is to present the results of the flood wave measurements from the precipitation of Cyclone Boris at the wellsensored experimental catchment (~36 km$^2$) of the Ludovika University of Public Service, Faculty of Water Sciences, near village of Magyaregregy.

Cyclone Boris brought significant precipitation to the whole of Europe, and the situation was no different in the Mecsek mountains of southern Hungary where Magyaregregy lies. It should be mentioned that there had been a long summer drought until the arrival of the cyclone. So, this significant amount of precipitation fell on a very dry catchment. As part of the precipitation-runoff study at the university, we also carried out flood wave and soil moisture measurements on the watershed during this precipitation event. We would like to present the results of these measurements in our poster.

Acknowledgments:
The research presented in the article was carried out within the framework of the
Széchenyi Plan Plus program with the support of the RRF 2.3.1 21 2022 00008
project.

Keywords: runoff, runoff test, runoff study, flood wave, drought, drought in the forest, precipitation
\newpage{}
\phantomsection
\stepcounter{articleid}
\addcontentsline{toc}{subsection}{Paulíková et~al. comparison of runoff coefficient estimations using a rational equation and statistical processing of rainfall-runoff data}
\begin{flushleft}

\abstrtitle{comparison of runoff coefficient estimations using a rational equation and statistical processing of rainfall-runoff data}

\name{Lynda Paulíková$^1$, Silvia Kohnová$^1$, Zuzana Danáčová$^2$, Milan Onderka$^{2,3}$}

\index{Paulíková, Lynda|textit}
\index{Kohnová, Silvia}
\index{Danáčová, Zuzana}
\index{Onderka, Milan}
\institute{$^1$ Slovak University of Technology in Bratislava (Slovakia)}

\institute{$^2$ Slovak Hydrometeorological Institute, Jeséniová 17, Bratislava, Slovakia SK-83315}

\institute{$^3$ Earth Science Institute SAS; Dept. of Atmospheric Physics, Dúbravská cesta 9, Bratislava, Slovakia SK-840 05}

\email{Corresponding author: lynda.paulikova@stuba.sk}

\end{flushleft}

\noindent

Runoff coefficients continue to be the subject of many discussions in hydrology. Despite having equations and tables to determine them, choosing the correct value is left to the user. Why? The physical processes involved in the runoff of a small catchment are not entirely understood. Consequently, final estimates of runoff coefficients may not match actual data.

This study aims to compare two methods for estimating the runoff coefficient in small catchments in Slovakia, with a return period of N = 100 years. The first part of the analysis employs a rational equation to estimate the runoff coefficient for 34 small catchments, using the Kirpich (1940) and Nash (1960) formulas to estimate the time of concentration. The second part statistically processes 33 years of precipitation and runoff data for two catchments: Horné Orešany – the Parná stream and Liptovský Hrádok – the Belá stream. The study is trying to assess the extent to which individual estimates differ and discuss the possibilities for the correct estimation of peak runoff coefficients on small catchments. 

Acknowledgements: The study was supported by VEGA Grant No. 1/0782/21; VEGA Grant No. 1/0577/23; APVV 23-0332 and APVV-20-0374.

Keywords: runoff coefficient, runoff, small catchment, flood waves

\newpage{}
\phantomsection
\stepcounter{articleid}
\addcontentsline{toc}{subsection}{Szeles et~al. Comparison of Two Isotopic Hydrograph Separation Methods in the Hydrological Open Air Laboratory}
\begin{flushleft}

\abstrtitle{Comparison of Two Isotopic Hydrograph Separation Methods in the Hydrological Open Air Laboratory}

\name{Borbala Szeles$^1$, Ladislav Holko$^2$, Juraj Parajka$^1$, Christine Stumpp$^3$, Michael Stockinger$^3$, Stefan Wyhlidal$^4$, Katharina Schott$^5$, Patrick Hogan$^1$, Lovrenc Pavlin$^6$, Peter Strauss$^7$, Günter Blöschl$^1$}

\index{Szeles, Borbala|textit}
\index{Holko, Ladislav}
\index{Parajka, Juraj}
\index{Stumpp, Christine}
\index{Stockinger, Michael}
\index{Wyhlidal, Stefan}
\index{Schott, Katharina}
\index{Hogan, Patrick}
\index{Pavlin, Lovrenc}
\index{Strauss, Peter}
\index{Blöschl, Günter}
\institute{$^1$ Institute of Hydraulic Engineering and Water Resources Management, Vienna University of Technology, Vienna, Austria}

\institute{$^2$ Institute of Hydrology, Slovak Academy of Sciences, Slovakia}

\institute{$^3$ University of Natural Resources and Life Sciences, Department of Water, Atmosphere and Environment, Institute of Soil Physics and Rural Water Management, Vienna, Austria}

\institute{$^4$ NES, Nuclear Engineering Seibersdorf GmbH, Austria}

\institute{$^5$ University of Natural Resources and Life Sciences, Institute of Soil Science, Stable Isotope Group, Tulln, Austria}

\institute{$^6$ Austrian Federal Ministry of Agriculture, Forestry, Regions and Water Management, Department for Water Balance, Austria}

\institute{$^7$ Federal Agency of Water Management, Institute for Land and Water Management Research, Austria}

\email{Corresponding author: szeles@hydro.tuwien.ac.at}

\end{flushleft}

\noindent

Exploring the contributions of runoff components during rainfall events in agricultural catchments is crucial for understanding runoff generation, solute transport, and soil erosion. The aim of this study was to compare two isotope hydrograph separation methods in a 66-hectare agricultural catchment in Austria at the Hydrological Open Air Laboratory (HOAL). The classical two-component (IHS) and ensemble hydrograph separation (EHS) were applied to multiple large events in May-October of 2013–2018 using $\delta^{18}\textrm{O}$ and $\delta^2\textrm{H}$. The results suggest that EHS provided average new water fractions during peak flows ($0.46\pm0.04\ \textrm{for}\ \delta^{18}\textrm{O},\ 0.47\pm0.03\ \textrm{for}\ \delta^2\textrm{H}$), which were close to the averages obtained by IHS ($0.47 \textrm{ for } \delta^{18}\textrm{O}, 0.50$ for $\delta^2\textrm{H}$). Large new water fractions during peak flows could be explained by the agricultural land use and soils with low permeability that promote overland flow generation and by some of the tile drainage systems contributing to the delivery of water. While EHS may be a more robust approach compared to IHS, as it relaxes some of the assumptions of IHS, IHS can provide information on the variability of new water contributions of individual events.

Keywords: stable isotopes of hydrogen and oxygen in water, isotopic hydrograph separation, experimental catchment, agricultural catchment, rainfall-runoff events

\newpage{}
\phantomsection
\stepcounter{articleid}
\addcontentsline{toc}{subsection}{Liová et~al. Multivariate analysis of seasonal flood waves: a case study of the Liptovská Mara Reservoir}
\begin{flushleft}

\abstrtitle{Multivariate analysis of seasonal flood waves: a case study of the Liptovská Mara Reservoir}

\name{Anna Liová$^1$, Roman Výleta$^1$, Kamila Hlavčová$^1$}

\index{Liová, Anna|textit}
\index{Výleta, Roman}
\index{Hlavčová, Kamila}
\institute{$^1$ Slovak University of Technology, Faculty of Civil Engineering, Department of Land and Water Resources Management}

\email{Corresponding author: anna.liova@stuba.sk}

\end{flushleft}

\noindent

In recent decades, the impact of climate change on river hydrology, particularly as a result of more intense and frequent flood discharges, has become widely discussed. Consequently, accurate flood data for designing and evaluating water infrastructures, including reservoirs, is now a priority. This study introduces a methodology for constructing control flood hydrographs used in flood risk mapping and dam safety assessments. It relies on a multivariate statistical analysis of flood wave characteristics from observed river discharge time series.

The study focuses on analysing discharge waves in the Liptovská Mara reservoir catchment, with a focus on seasonal flood wave characteristics. Two inflow estimation methods were compared: (i) a simplified approach using discharge increases from the main stream, and (ii) a comprehensive approach integrating inflows from the main stream and its tributaries. Flood events from 1989 to 2021 were identified using FloodSep software, which separated the flood waves into annual maximums and seasonal peaks for the summer and winter. 

The methodology reveals seasonal variations in flood characteristics, i.e., winter flood waves are longer due to snowmelt, while summer waves are shorter, because they are driven by storms. These insights support more precise seasonal flood risk management.

Acknowledgements: This study was financially supported by the Slovak Research and Development Agency under Contract No. APVV 23-0332. 

Keywords: discharge waves, flood wave separation, seasonality


\newpage{}
\phantomsection
\stepcounter{articleid}
\addcontentsline{toc}{subsection}{Khalil and Parajka. Improving the accuracy of hydrological models through integration of satellite data in alpine and lowland regions}
\begin{flushleft}

\abstrtitle{Improving the accuracy of hydrological models through integration of satellite data in alpine and lowland regions}

\name{Asma Khalil$^1$, Juraj Parajka$^1$}

\index{Khalil, Asma|textit}
\index{Parajka, Juraj}
\institute{$^1$ Vienna University of Technology, Institute of Hydraulic Engineering and Water Resources Management }

\email{Corresponding author: Asma Khalil, khalil@hydro.tuwien.ac.at}

\end{flushleft}

\noindent

Hydrological models are essential for analysing and forecasting water flow patterns across varied geographic landscapes. However, traditional modelling techniques often face limitations, especially in areas with limited ground-based observations. This study explores the spatial and temporal performance of the TUW hydrological model during its calibration and validation and highlights shortcomings in conventional modelling. The effectiveness of integrating satellite-derived data is assessed, specifically ASCAT soil moisture and MODIS snow cover, to enhance model performance in data-scarce regions. While prior studies, such as Tong et al. (2022), have conducted multi-objective calibrations using runoff, soil moisture, and snow data, they did not address seasonal calibration. Our results show that incorporating satellite data significantly improves the TUW model's predictive performance, particularly in the Alpine regions. Although improvements in some lowland catchments were more modest, overall findings emphasize the benefits of satellite data in enhancing the accuracy of hydrological modelling. This research underscores the potential of satellite-derived data to bridge gaps in traditional hydrological modelling in data-limited settings, thereby offering valuable insights for better water resource management and flood forecasting in diverse landscapes.

Keywords: Hydrological modelling, satellite data, TUW model, ASCAT soil moisture, MODIS snow cover, data-scarce regions

\newpage{}
\phantomsection
\stepcounter{articleid}
\addcontentsline{toc}{subsection}{Vigoureux et~al. Feedbacks on a comparison of flows estimated by image analysis (LSPIV) and by hydrological and hydraulic modelling on the Côte d'Azur during the flash flood of November 2019}
\begin{flushleft}

\abstrtitle{Feedbacks on a comparison of flows estimated by image analysis (LSPIV) and by hydrological and hydraulic modelling on the Côte d'Azur during the flash flood of November 2019}

\name{Sarah Vigoureux$^1$, Léa-Linh Liebard$^1$, Aubin Chonoski$^1$, Etienne Robert$^1$, Louis Torchet$^1$, Valentin Poveda$^1$, Frédérique Leclerc$^2$, Jérémy Billant$^2$, Rémi Dumasdelage$^3$, Gauthier Rousseau$^{4,6}$, Olivier Delestre$^5$, Pierre Brigode$^{2,6}$}

\index{Vigoureux, Sarah|textit}
\index{Liebard, Léa-Linh}
\index{Chonoski, Aubin}
\index{Robert, Etienne}
\index{Torchet, Louis}
\index{Poveda, Valentin}
\index{Leclerc, Frédérique}
\index{Billant, Jérémy}
\index{Dumasdelage, Rémi}
\index{Rousseau, Gauthier}
\index{Delestre, Olivier}
\index{Brigode, Pierre}
\institute{$^1$ Polytech Nice Sophia, Université Côte d’Azur, France }

\institute{$^2$ Université Côte d’Azur, CNRS, OCA, IRD, Géoazur, Nice, France }

\institute{$^3$ Métropole Nice Côte d’Azur and Nice Municipality, Nice, France}

\institute{$^4$ Institute of Hydraulic Engineering and Water Resources Management, TU Wien, Karlsplatz 13, 1040 Vienna, Austria}

\institute{$^5$ Université Côte d’Azur, CNRS, LJAD, Nice, France }

\institute{$^6$ Univ Rennes, CNRS, Géosciences Rennes - UMR 6118, Rennes F-35000, France}

\email{Corresponding author: gauthier.rousseau@tuwien.ac.at}

\end{flushleft}

\noindent

Short and heavy rainstorm events often lead to flash floods on the French Riviera coastal catchments: during the historical flood of 2 October 2015, a peak streamflow value between 185 and 295 m$^3$/s was estimated on the Brague River at Biot at 10:30 PM, while the streamflow was around 1 m$^3$/s at 6:30 PM at the same section. If the measurements of such streamflow values are highly important (for the statistical analysis and modelling of floods and hydraulic structure design), such measurements are dangerous when they require an operator to manipulate an instrument in or near a river. Alternative methods, such as video analysis, can be used by analyzing a sequence of images and locating the displacement of patterns on the water surface. Thus, the velocity field at the surface of the flow can be determined and then used for estimating flow discharges on specific cross sections.  In this work, we applied two different Large-Scale Particle Image Velocimetry (LSPIV) algorithms, i.e., Fudaa-LSPIV and OpyFlow, to several videos of the November 2019 floods within the Brague catchment in order to estimate the streamflow values. The streamflow values were then compared to the values estimated through the observations available, and also to the results of (i) rainfall-runoff modeling and (ii) hydraulic modeling on the same sections. The LSPIV estimations, rainfall-runoff simulations, and observations are coherent on the sections studied, thereby showing the benefit of combining such different and independent techniques in order to estimate flood streamflow values.

Keywords: Gauging, hydrometry, LSPIV, hydraulic, hydrological, modelling, flash flood

\newpage{}
\phantomsection
\stepcounter{articleid}
\addcontentsline{toc}{subsection}{Koch et~al. Results of the first measurements on the interception garden in Baja}
\begin{flushleft}

\abstrtitle{Results of the first measurements on the interception garden in Baja}

\name{Dániel Koch$^1$, Balázs Abonyi$^1$, Gábor Keve$^1$, Péter Kalicz$^2$, András Herceg$^2$, Zoltán Gribovszki$^2$}

\index{Koch, Dániel|textit}
\index{Abonyi, Balázs}
\index{Keve, Gábor}
\index{Kalicz, Péter}
\index{Herceg, András}
\index{Gribovszki, Zoltán}
\institute{$^1$ Faculty of Water Sciences, Ludovika University of Public Service, Hungary}

\institute{$^2$ Faculty of Forestry, University of Sopron, Hungary}

\email{Corresponding author: koch.daniel@uni-nke.hu}

\end{flushleft}

\noindent

In August 2024, the Department of Regional Water Management of the Faculty of Water Sciences of the Ludovika University of Public Service, Hungary, undertook an important role in an international research project aimed at investigating the rainfall retention capacity of urban vegetation. The OTKA project entitled “Microscale influence on runoff” will run until the end of 2025 and involves collaboration with the University of Ljubljana and the University of Sopron.

The project will focus on the interactions between vegetation and precipitation, which are also key for water management in urban environments. The aim of the research is to investigate the efficiency of inland vegetation in retaining precipitation and how it affects the water cycle in urban areas.

One of the highlights of our faculty's involvement is the interception garden at the Baja Campus of the Ludovika University of Public Service. The measuring stations installed here were set up on 13 August 2024 with the help of staff from the University of Sopron. 

A plot was planted in the park, where maple trees (Acer sp.) are planted in a 10*10 m network. We measure the meteorological characteristics (precipitation, wind direction and speed, humidity, temperature, soil moisture, etc.), especially the precipitation that falls through the canopy and the stem runoff. In addition to the manual measurements, we also set automatic measurements with a frequency of 10 minutes.

The data collected will be compared with the results from Ljubljana and Sopron to find empirical correlations that will help describe the phenomenon in engineering terms. The results of the project will make a significant contribution to improving urban water management practices and to understanding and managing the impacts of climate change better.

The poster shows the initial precipitation events and the corresponding measurement results.

Acknowledgement: This contribution is part of ongoing research entitled “Microscale influence on runoff” supported by the Slovenian Research and Innovation Agency (N2-0313) and National Research, Development, and Innovation Office (OTKA Project No. SNN143972).  

Keywords: iterception, precipitation, through fall, experimental plot
\newpage{}
\phantomsection
\stepcounter{articleid}
\addcontentsline{toc}{subsection}{Zaťovičová and Majorošová. Optimizing a Green-Blue Infrastructure: Microclimatic Analysis and Adaptive Measures for Mitigating Urban Heat Islands}
\begin{flushleft}

\abstrtitle{Optimizing a Green-Blue Infrastructure: Microclimatic Analysis and Adaptive Measures for Mitigating Urban Heat Islands}

\name{Miriam Zaťovičová$^1$, Martina Majorošová$^1$}

\index{Zaťovičová, Miriam|textit}
\index{Majorošová, Martina}
\institute{$^1$ Slovak University of Technology, Department of Land and Water Resources Management}

\email{Corresponding author: miriam.zatovicova@stuba.sk}

\end{flushleft}

\noindent

Climate change is a significant threat and the necessary reduction in temperatures to ameliorate it is unlikely to be achieved. Urbanisation, which leads to an increase in impermeable surfaces, causes the overheating of cities and contributes to the formation of urban heat islands (UHIs). Rapidly growing cities are exposed to climate change, which is exacerbated by paved surfaces that lead to higher temperatures and the need for more cooling. An effective UHI solution involves the use of a blue-green infrastructure to provide shade and cool the atmosphere through transpiration. However, several factors need to be considered when implementing adaptation measures, such as the correct orientation of buildings and ensuring sufficient airflow. Bratislava does not yet have a coherent methodology for adapting to climate change. Projects are often isolated and do not cover the need for a comprehensive blue-green solution. An analysis of the microclimatic conditions of the Old Town in Bratislava can help in designing effective measures to overcome the challenges of climate change.

Acknowledgements: This study was financially supported by the VEGA Grant Agency No. VEGA 1/0067/23.

Keywords: Urban heat island mitigation, climate resilient city, adaptation strategy
\newpage{}
\phantomsection
\stepcounter{articleid}
\addcontentsline{toc}{subsection}{Grečnárová et al. The effect of substrate depth on stormwater retention in vegetated roofs}
\begin{flushleft}

\abstrtitle{The effect of substrate depth on stormwater retention in vegetated roofs}

\name{Jana Grečnárová$^1$, Michaela Danáčová$^1$, Matúš Tomaščík$^1$}

\index{Grečnárová, Jana|textit}
\index{Danáčová, Michaela}
\index{Matúš Tomaščík}
\institute{$^1$ Slovak University of Technology, Slovakia, Faculty of Civil Engineering, Department of Land and Water Resources Management }

\email{Corresponding author: jana.grecnarova@stuba.sk}

\end{flushleft}

\noindent

This study aimed to demonstrate the effect of the substrate in a vegetated roof on reducing stormwater runoff. The experimental approach involved measurements at different rainfall intensities and substrate depths, ensuring identical initial conditions for all the measurements. The substrate moisture content was up to a maximum value of 5\%. Each measurement was repeated three times and tested using the same rainfall combinations.

The results indicated that the height of a substrate did not significantly affect water retention at lower rainfall intensities. More pronounced differences in water retention were observed at higher rainfall intensities, where the effect of the substrate depth was also evident. The time course of the runoff showed the highest increase in volume after 9 minutes.

At higher rainfall intensities (2.7 and 3.8 mm/min), the substrate retention increased 8–16\% at 7 and 10 cm depths, while retention increased by 12\% when comparing the 10 and 14 cm depths. At lower rainfall intensities, the difference in the effect of the substrate depth was not significant.

Acknowledgements: The study was supported by the VEGA grant project No. 1/0577/23 and APVV 20-0374.

Keywords: rainfall intensity, substrate depth, vegetated roof, runoff
\newpage{}
\phantomsection
\stepcounter{articleid}
\addcontentsline{toc}{subsection}{Ogunfolaji et~al. Analysis of throughfall characteristics under a pine tree canopy in different climate settings: A case study of Slovenia and Hungary}
\begin{flushleft}

\abstrtitle{Analysis of throughfall characteristics under a pine tree canopy in different climate settings: A case study of Slovenia and Hungary}

\name{Yusuf Oluwasegun Ogunfolaji$^1$, Mark Bryan Alivio$^1$, Kamilla Orosz$^2$, András Herceg$^2$, Péter Kalicz$^2$, Katalin Anita Zagyvai-Kiss$^2$, Zoltán Gribovszki$^2$, Nejc Bezak$^1$}

\index{Ogunfolaji, Yusuf|textit}
\index{Alivio, Mark}
\index{Orosz, Kamilla}
\index{Herceg, András}
\index{Kalicz, Péter}
\index{Zagyvai-Kiss, Katalin}
\index{Gribovszki, Zoltán}
\index{Bezak, Nejc}
\institute{$^1$ University of Ljubljana, Faculty of Civil and Geodetic Engineering, Ljubljana, Slovenia}

\institute{$^2$ University of Sopron, Institute of Geomatics and Civil Engineering, Hungary}

\email{Corresponding author: yogunfol@fgg.uni-lj.si}

\end{flushleft}

\noindent

Trees can be an important element of urban greenery and a nature-based solution to mitigate natural
hazards such as floods. Therefore, knowledge about rainfall interception characteristics is vital for
adequately planning urban greenery. Moreover, due to the role of trees in altering the hydrological
cycle, it is essential to understand how factors such as leaf cover, meteorological variables and their
seasonal variations, and the magnitude of rainfall affect throughfall under a pine tree canopy in
different climatic settings. Hence, the objective of this study was to compare the throughfall
characteristics of two black pine trees (Pinus nigra) positioned in urban parks. To achieve the aim of
this study, we analyzed open-air rainfall (i.e., gross rainfall) and rainfall falling through a pine tree
canopy (i.e., throughfall) from two research experimental sites located in the city of Ljubljana,
Slovenia, and the University of Sopron’s Botanical Garden, Hungary. These sites were referred to as
Plots 1 and 2, respectively. The measurement period used within this study ranged from September
2023 to September 2024. Manual and automatic precipitation measurements were conducted at both
sites.

Based on the results of this study, the mean throughfalls over the
measurement period were 45\% and 50\% of the gross rainfall for Plots
1 and 2, respectively. The relationship between the gross rainfall and
throughfall for each case study was established using linear
regression, and the results showed a strong correlation with and for
Plots 1 and 2, respectively, and were statistically significant with a
p-value < 0.01 for both case studies.

To further compare the throughfall between the two case studies, we divided the gross rainfall and
throughfall event based on the phenological season (i.e., leafless and leafed periods), the calendar
seasons of the year, and the magnitude of the rainfall. According to the leaf cover classification, the
average throughfalls for plot 1 were 41\% and 46\% of the gross rainfall during the leafed and leafless
periods, whereas, for plot 2, the average throughfalls were 49\% and 51\%, respectively. For both
plots, seasonal analysis showed consistent trends across the autumn, spring, and winter, but
differences emerged in the summer. Although the average throughfall for plot 1 varied between 37\%
and 52\%, while that of plot 2 ranged from 42\% to 54\%.

Lastly, the rainfall events were grouped based on the amount of rainfall into five different classes, and
it was observed that for both plots, the percentage of throughfall increased as the magnitude of the
rainfall increased. In the following steps of the study, the impact of the meteorological variables on
the throughfall characteristics of both sites will be investigated.

Acknowledgment: This study is part of ongoing research entitled “Microscale influence on runoff” supported by the Slovenian Research and Innovation Agency (N2-0313) and National Research, Development, and Innovation Office (OTKA project grant number SNN143972). The work was also supported through the Ph.D. grant of the first author, which is financially supported by the Slovenian Research and Innovation Agency.

Keywords: pine tree, urban greenery, nature-based solutions, rainfall interception, meteorological variables
\newpage{}
\phantomsection
\stepcounter{articleid}
\addcontentsline{toc}{subsection}{Brusasco et~al. Soil erosion in agricultural landscapes: a WEPP model case study at the Hydrological Open Air Laboratory in Petzenkirchen, Austria}
\begin{flushleft}

\abstrtitle{Soil erosion in agricultural landscapes: a WEPP model case study at the Hydrological Open Air Laboratory in Petzenkirchen, Austria}

\name{Luca Brusasco$^1$, Ilaria Gnecco$^1$, Anna Palla$^1$, Giorgio Roth$^1$, Günter Blöschl$^2$}

\index{Brusasco, Luca|textit}
\index{Gnecco, Ilaria}
\index{Palla, Anna}
\index{Roth, Giorgio}
\index{Blöschl, Günter}
\institute{$^1$ University of Genova, Genoa, Italy, Department of Civil, Chemical and Environmental Engineering}

\institute{$^2$ TU Wien, Vienna, Austria, Institute of Hydraulic Engineering and Water Resources Management E222}

\email{Corresponding author: 4518846@studenti.unige.it}

\end{flushleft}

\noindent

Soil erosion poses significant environmental risks that can lead to land degradation, reduced agricultural productivity, and impaired water quality. This research models soil erosion processes within a 66-hectare agricultural catchment at the Hydrological Open Air Laboratory (HOAL) in Petzenkirchen, Austria, using the Water Erosion Prediction Project (WEPP) model. WEPP is a process-based, spatially-distributed model that simulates soil detachment, transport, and deposition on both hillslope and watershed scales to offer detailed predictions of sediment generation.

The study focuses on single-storm simulations of historical rainfall events to assess the model’s accuracy in predicting erosion. Sensitivity analysis is performed to understand the effect of key parameters on erosion rates. Although the focus is currently on single events, WEPP’s ability to run continuous simulations presents opportunities for future exploration of long-term erosion trends, including the effects of climate change and seasonality on runoff and sediment generation.

Land-use scenario simulations are also conducted to evaluate their impact on erosion dynamics to provide insights for sustainable land and water management. This research seeks to identify critical erosion zones and support effective land management practices.

Keywords: soil erosion, WEPP, GeoWEPP, hydrological modelling, land management, climate change, land use change
\newpage{}
\phantomsection
\stepcounter{articleid}
\addcontentsline{toc}{subsection}{Thoma et~al. Identification of critical source areas for sediment erosion (and  phosphorus loss) in a small agricultural catchment }
\begin{flushleft}

\abstrtitle{Identification of critical source areas for sediment erosion (and  phosphorus loss) in a small agricultural catchment }

\name{Christopher Thoma$^{1,2}$, Elmar Schmaltz$^3$, Borbala Szeles$^{1,2}$, Carmen Krammer$^3$, Peter Strauss$^3$, Günter Blöschl$^{1,2} $}

\index{Thoma, Christopher|textit}
\index{Schmaltz, Elmar}
\index{Szeles, Borbala}
\index{Krammer, Carmen}
\index{Strauss, Peter}
\index{Blöschl, Günter}
\institute{$^1$ Institute of Hydraulic Engineering and Water Resources Management, Vienna  University of Technology, Vienna, Austria }

\institute{$^2$ Centre for Water Resource Systems, Vienna University of Technology, Vienna,  Austria }

\institute{$^3$ Institute for Land and Water Management Research, Federal Agency for Water  Management, Petzenkirchen, Austria }

\email{Corresponding author: thoma@hydro.tuwien.ac.at }

\end{flushleft}

\noindent

This study focuses on addressing soil erosion and phosphorus (P) runoff, critical issues for  agricultural sustainability and water quality. Conducted in the Hydrological Open Air  Laboratory (HOAL) in Petzenkirchen, Austria, the research aims to identify fields prone to  erosion and predict future sediment and P loss under different scenarios. The 60-hectare HOAL  catchment, which is typical of Austria’s Alpine foothills, offers diverse land use and extensive historical  data, making it ideal for such investigations. 

Using data from 2010 to 2020, the study examines factors like soil management, fertilizer use,  and erosive rainfall events. Advanced geospatial and statistical techniques, including GIS tools  and regression models, will help map erosion-prone areas and identify key drivers of sediment  and P loss. 

The research aims to produce risk maps that will be able to inform land management decisions, helping to  reduce sediment and P loss from high-risk fields. These findings will support sustainable  agricultural practices and be useful for policymakers, farmers, and environmental scientists  working to balance productivity with environmental protection. 

Keywords: Soil Erosion, P loss, Water Quality, Sustainable Agriculture, Erosion  Susceptibility

\newpage{}
\phantomsection
\stepcounter{articleid}
\addcontentsline{toc}{subsection}{Bajtek et~al. The impact of a reservoir on the water temperature of the Ipel River}
\begin{flushleft}

\abstrtitle{The impact of a reservoir on the water temperature of the Ipel River}

\name{Zbyňek Bajtek$^1$, Veronika Bačová Mitková$^1$, Igor Leščešen$^1 $}

\index{Bajtek, Zbyňek|textit}
\index{Mitková, Veronika}
\index{Leščešen, Igor}
\institute{$^1$ Institute of Hydrology SAS, Dúbravská cesta č. 9, 841 04, Bratislava, Slovak Republic }

\email{Corresponding author: bajtek@uh.savba.sk }

\end{flushleft}

\noindent

This study examines the impact of reservoirs and climate change on the temperature regime of river systems. It is noted that cold water discharges from reservoirs can affect river temperature levels, thereby leading to decreases during warm weather and increases during the winter months. However, as warmer air temperatures and increased solar radiation become more pronounced in the future, the impact of reservoirs on downstream water temperatures may diminish. Nevertheless, reservoirs still play a significant role in cooling downstream river reaches that experience temperature stratifications. The study analyses data from climate stations and water temperature recording stations located over and downstream of reservoirs. The findings indicate a significant increase in water temperatures throughout the year, with a more pronounced increase in air temperatures during the warmer months. The presence of the reservoir has a noticeable effect on average monthly temperatures, resulting in a slight decrease in the summer and an increase in the winter. In our paper we build on the work of Gough, who developed a method to analyse the temperature of the water in a river and combine it with the influence of the reservoir. The assessment itself is divided into two parts according to the data available: in the first part, stations with data for the period before the construction of the reservoir are used. In the second phase, stations with shorter measurement periods are included in the assessment and compared with the first. This comparison showed that the reservoir affects the river water temperature throughout the year by cooling the river in the summer and warming it in the winter.
\newpage{}
\phantomsection
\stepcounter{articleid}
\addcontentsline{toc}{subsection}{Juhász et~al. Changes in water resources in the West-Transdanubian region of Hungary}
\begin{flushleft}

\abstrtitle{Changes in water resources in the West-Transdanubian region of Hungary}

\name{István Juhász$^{1,2}$, Zoltán Gribovszki$^2$, Éva Fruzsina Kapolcsi$^3$, Csilla Tamás$^3$, Péter Kalicz$^2$}

\index{Juhász, István|textit}
\index{Gribovszki, Zoltán}
\index{Kapolcsi, Éva}
\index{Tamás, Csilla}
\index{Kalicz, Péter}
\institute{$^1$ West-Transdanubian Water Directorate, Department of Water Protection and River Basin Management;}

\institute{$^2$ University of Sopron, Faculty of Forestry, Institute of Geomatics and Civil Engineering;}

\institute{$^3$ West-Transdanubian Water Directorate, Department of Hydrography and Data Archives;}

\email{Corresponding author: juhasz.istvan@nyuduvizig.hu}

\end{flushleft}

\noindent

As a result of climate change, the hydrological cycle has changed significantly in Hungary, and thus also in its west Transdanubian region, in recent decades. While the annual amount of precipitation has not changed significantly, the distribution of precipitation within the year has changed, and the distribution probability of intense precipitation events has increased. In addition, the length of rain-free periods and the average annual temperature have increased, which has lengthened the severity and duration of droughts. As a result, the intensity of evaporation has also increased, which together with the previous factors, has reduced the amount of surface water and groundwater resources.
The primary goal of the work is to quantify the reduction of water resources in watersheds with different climates, characteristics and sizes. It is important to determine the factors that influence the sensitivity of individual watersheds to the effects of climate change. If we know these factors, we can find the best measures with which water retention is the most effective, thus reducing the reduction of water resources.

Keywords: water resources, climate change, watershed characteristics

\newpage{}
\phantomsection
\stepcounter{articleid}
\addcontentsline{toc}{subsection}{Chappon and Bene. A data-driven approach aimed to close the joint water balance of Lake Velence and its two regulating reservoirs}
\begin{flushleft}

\abstrtitle{A data-driven approach aimed to close the joint water balance of Lake Velence and its two regulating reservoirs}

\name{Máté Chappon$^1$, Katalin Bene$^1 $}

\index{Chappon, Máté|textit}
\index{Bene, Katalin}
\institute{$^1$ National Laboratory for Water Science and Water Security, Széchenyi István University, Department of Transport Infrastructure and Water Resources Engineering Egyetem tér 1. H-9026 Győr, Hungary }

\email{Corresponding author: chappon.mate@sze.hu}

\end{flushleft}

\noindent

Lake Velence, Hungary's third-largest lake, is a nationally significant recreational and ecological site protected under the Ramsar Convention. However, a severe drop in water levels in September 2022, which was driven by climate conditions and human activities, highlighted underlying conflicts among water users around the lake and its watershed. This event has raised concerns about the ability of the current water management approach to address future challenges posed by climate change and increasing water demand. 

A data-driven approach was developed to refine the lake's interconnected water balance, including two upstream reservoirs to Lake Velence, and improve management. The water balance components in this system have been monitored for decades, yet errors in the data have created uncertainty in management decisions. Two multiple regression models were developed using monthly data from 1998 to 2022: the first one uses uniform regression coefficients across all the months for each water balance element, while the second adjusts the regression coefficients on a monthly basis for each water balance element to account for the seasonal variability in the error patterns.

In the final step, the residual errors were proportionally redistributed across the components, to achieve a closed water balance. This revised time series supports the creation of water management scenarios designed to meet target water levels across all three water bodies. These scenarios allow for a comprehensive comparison of each element of the ecosystem services, thereby offering valuable insights for sustainable water use. This approach aims to strengthen Lake Velence's resilience to ensure it meets recreational, ecological, and community needs in an increasingly uncertain climate future.
\newpage{}
\phantomsection
\stepcounter{articleid}
\addcontentsline{toc}{subsection}{Kovács et~al. Groundwater monitoring and water retention (a case study in the Hidegvíz Valley)}
\begin{flushleft}

\abstrtitle{Groundwater monitoring and water retention (a case study in the Hidegvíz Valley)}

\name{Júlia Kovács$^1$, András Herceg$^1$, Péter Kalicz$^1$, Katalin Anita Zagyvai-Kiss$^1$, Zoltán Gribovszki$^1 $}

\index{Kovács, Júlia|textit}
\index{Herceg, András}
\index{Kalicz, Péter}
\index{Zagyvai-Kiss, Katalin}
\index{Gribovszki, Zoltán}
\institute{$^1$ University of Sopron, Faculty of Forestry , Institute of Geomatics and Civil Engineering }

\email{Corresponding author: k.julia547@gmail.com}

\end{flushleft}

\noindent

Precipitation distribution has become more extreme in recent decades and result in more droughts than in the past. For some vegetation types, such as a hygrophite intrazonal alder forest, surplus water is essential. Therefore, we investigated the effects of a temporary water retention. The study area is located in the Sopron Hills, in the Hidegvíz Valley. Detailed groundwater monitoring has been in operation for more than a decade in the vicinity of temporary water retention. In the context of this study, this network was extended (to 21 groundwater wells), and some wells were equipped with automatic water level recorders. Six and a half years of manual measurements were processed; automatic data loggers recorded the water table for one month after the establishment of the log weir. The water retention caused a 40 cm rise in the streamwater level and a several decimeter (dm) rise of the groundwater table in the vincinity of the streambed (3–4 m) within a few hours. This increase in the groundwater levels provided a more favourable groundwater table for the alder forest along the stream during the dry months of July and August. 

Keywords: water table, groundwater supply, alder forest, log weir
\newpage{}
\phantomsection
\stepcounter{articleid}
\addcontentsline{toc}{subsection}{Dobó et~al. Comparison of the groundwater uptake habits of lowland pedunculate oak (Quercus robur l.) and poplar (Populus x) stands, with special reference to the depth limit of the groundwater access}
\begin{flushleft}

\abstrtitle{Comparison of the groundwater uptake habits of lowland pedunculate oak (Quercus robur l.) and poplar (Populus x) stands, with special reference to the depth limit of the groundwater access}

\name{Márton Dobó$^1$, Bence Bolla$^1$, András Szabó$^1$}

\index{Dobó, Márton|textit}
\index{Bolla, Bence}
\index{Szabó, András}
\institute{$^1$ University of Sopron, Forest Research Institute }

\email{Corresponding author: dobo.marton@uni-sopron.hu}

\end{flushleft}

\noindent

Water regulation works and the increasing water usage since then have led to a decline in groundwater levels on the Great Hungarian Plain. As a result, the habitat areas suitable for groundwater-dependent tree species are continuously shrinking. This ongoing and unfavourable trend seriously raises issues concerning the long-term sustainability of the forest stands affected. The goal of this paper is to determine hydrological thresholds through the study of various sample areas in order to enable more accurate planning of these forested areas. It investigates the groundwater use of pedunculate oak and hybrid poplar stands at the groundwater monitoring sites selected.

The research aims to examine and compare the groundwater uptake habits of pedunculate oak and hybrid poplar stands, with particular attention paid to the depth limit of the accessible groundwater. Additionally, the paper analyses the groundwater level dynamics in the sample areas. The research will be based on groundwater level time series measured at a high temporal frequency. The process of selecting the sample areas is currently underway.

Keywords: Great Hungarian Plain, groundwater uptake, groundwater monitoring, accessibility of groundwater 

\newpage{}
\phantomsection
\stepcounter{articleid}
\addcontentsline{toc}{subsection}{Kálmán et~al. The Importance of Economic Evaluation in Water Management Decisions – A Case Study in the Watershed of Lake Velence}
\begin{flushleft}

\abstrtitle{The Importance of Economic Evaluation in Water Management Decisions – A Case Study in the Watershed of Lake Velence}

\name{Attila Kálmán$^1$, Máté Chappon$^1$, Katalin Bene$^1$}

\index{Kalman, Attila@K\'{a}lm\'{a}n, Attila|textit}
\index{Chappon, Máté}
\index{Bene, Katalin}
\institute{$^1$ National Laboratory for Water Science and Water Security, Széchenyi István University, Department of Transport Infrastructure and Water Resources Engineering, Egyetem square 1., H-9026 Győr, Hungary}

\email{Corresponding author: kalman.attila@sze.hu}

\end{flushleft}

\noindent

Climate change has altered the water cycle in Hungary’s Central Transdanubia region, exacerbating water-related conflicts in the Lake Velence catchment area. As the number of water users has risen, so has the demand for water-related services, thereby exacerbating the severe water scarcity across the area.

A 10-minute online questionnaire was conducted to assess values associated with the lake and watershed, evaluate their state and inherent challenges to understand the perspectives of locals and visitors of the water services, and envision potential future scenarios. The survey also included questions on the valuation of water resources using the "willingness-to-pay" (WTP) methodology, which allowed for an economic assessment of water-related services across diverse stakeholder groups. 

Thanks to the 840 respondents, the findings provide a comprehensive analysis of the issues facing Lake Velence and its watershed and underscore the necessity of integrated water management at the catchment level to achieve sustainable outcomes. The study revealed a broad acknowledgment of the pressing water scarcity issues in the region, with a significant portion of stakeholders expressing a willingness to contribute financially to improve the quality and availability of the water and related ecosystem services. 

By combining strategic design approaches with an economic evaluation, the study identifies actionable and sustainable pathways for developing and supporting policy recommendations and community-driven initiatives. The approach emphasizes the ecological and social value of water resources. It presents an opportunity for more resilient water management that aligns with the economic and environmental priorities in the Lake Velence catchment area.

Keywords: sustainability, integrated water management, Lake Velence, WTP methodology, economic evaluation
\newpage{}
\phantomsection
\stepcounter{articleid}
\addcontentsline{toc}{subsection}{Farooq et~al. Comparative analysis of machine learning and statistical models for enhanced wet-season rainfall forecasting in Northern Australia}
\begin{flushleft}

\abstrtitle{Comparative analysis of machine learning and statistical models for enhanced wet-season rainfall forecasting in Northern Australia}

\name{Rashid Farooq$^{1,2}$, Monzur Imtiaz$^1$, Kamila Hlavčová$^2$, Silvia Kohnová$^2 $}

\index{Farooq, Rashid|textit}
\index{Imtiaz, Monzur}
\index{Hlavčová, Kamila}
\index{Kohnová, Silvia}
\institute{$^1$ Department of Civil \& Construction Engineering, Swinburne University of Technology, Melbourne 3122, Australia. }

\institute{$^2$ Department of Land and Water Resources Management, Faculty of Civil Engineering, Slovak University of Technology, Bratislava, Slovak Republic. }

\email{Corresponding author: 102055073@student.swin.edu.au}

\end{flushleft}

\noindent
 
Accurately predicting seasonal rainfall variations in the Northern Territory (NT) is crucial for effective water resource management. These variations are highly complex and influenced by several significant climatic anomalies. In this study, both supervised machine learning (ML) and statistical models were used to analyze seasonal rainfall patterns at two distinct stations in the NT, Australia by drawing on a monthly dataset from 1900 to 2023. The research aims to uncover the nonlinear relationship between the NT's wet-season rainfall and lagged climate indices by applying the Seasonal Autoregressive Integrated Moving Average (SARIMA) model and the CatBoost ML model, with input datasets incorporating various climate indices. Comparative assessments were conducted using performance metrics such as the root mean square error (RMSE), mean absolute error (MAE), and correlation coefficient (R) to reveal a marked improvement in the model performance with the statistical approach. The results demonstrated that the ML model produced higher correlation and less error by incorporating lagged indices to forecast wet-season rainfall, thereby outperforming the SARIMA method. In the testing phase, the CatBoost model’s correlation achieved 83\% and 77\% for the Alexandria and Anthony Lagoon stations in NT, respectively. 
 
Keywords: Climate indices, CatBoost, Machine learning, Rainfall, SARIMA
\newpage{}
