\phantomsection
\stepcounter{articleid}
\addcontentsline{toc}{subsection}{Onderka and Garaj. Maps of area-averaged rainfall intensities across Slovakia}
\begin{flushleft}

\abstrtitle{Maps of area-averaged rainfall intensities across Slovakia}

\name{Milan Onderka$^1$, Marcel Garaj$^1$}

\index{Onderka, Milan|textit}
\index{Garaj, Marcel}
\institute{$^1$ Slovak Hydrometeorological Institute, Jeséniova 17, Bratislava SK-833 15, Slovakia }

\email{Corresponding author: milan.onderka@shmu.sk}

\end{flushleft}

\noindent

The ”Rational Method” is a hydrological technique used for estimating peak discharges based on rainfall intensity, a catchment's runoff coefficient, and its area. This study aimed to create vector maps of area-averaged rainfall intensities as the input for the Rational Method for all of Slovakia. A vector map with over four thousand small catchments in Slovakia was used to calculate catchment-specific rainfall frequencies. The rainfall frequencies were determined using the annual maximum series (AMS), with parameters of the Generalized Extreme Value distribution inferred through a Bayesian approach. Rainfall frequencies were estimated for durations from 5 minutes to 6 hours and return periods from 1 to 100 years. Quantiles corresponding to return periods < 10 years were adjusted using Langbein’s formula. Local estimates of rainfall frequencies were spatially interpolated to produce accurate estimates for areas without in-situ rainfall measurements. We employed tri-variate regularized splines with tension to facilitate this interpolation and incorporated the altitude as an additional covariate to the geographical locations. The resulting maps of the area-averaged rainfall intensities are now available for all the small catchments in Slovakia.

Keywords: rainfall, intensity, rational method, frequency analysis
\newpage{}
\phantomsection
\stepcounter{articleid}
\addcontentsline{toc}{subsection}{Füssi-Nagy et~al. Analysis of changes in convective precipitation by comparing climate model data to observation results in Hungary}
\begin{flushleft}

\abstrtitle{Analysis of changes in convective precipitation by comparing climate model data to observation results in Hungary}

\name{Regő Füssi-Nagy$^1$, Balázs Sándor$^1$, László Báder$^1$}

\index{Füssi-Nagy, Regő|textit}
\index{Sandor, Balázs@S\'{a}ndor, Bal\'{a}zs}
\index{Bader, Laszlo@B\'{a}der, L\'{a}szl\'{o}}
\institute{$^1$ Budapest University of Technology and Economics, Department of Hydraulic and Water Resources Engineering }

\email{Corresponding author: fussi-nagy.rego@edu.bme.hu fussinagy.rego@gmail.com}

\end{flushleft}

\noindent

The impact of climate change increases surface temperatures and the concentration of water vapour in the atmosphere. The latter is further increased by the instability of an air column and vertical advection. As a result, convective precipitation (local rainfall) is increasing more than expected, based on the Clausius-Clapeyron equation. 

We have analysed monthly averaged data of the 0.25° resolution ERA5 reanalysis model in the Danube Basin for the following parameters: convective and large-scale precipitation. The findings were compared to point observations data from 4 local Hungarian SYNOP meteorological sites from 2000 to 2015. We have found a correlation coefficient of ~0.9 between the two types of convective precipitation, with a significance level of p<0.005. The validated results motivated us to extend the scope of the analysis. Correspondences were searched between the previous and the following parameters: soil moisture content, air temperature at 2m, and vertical velocity.

Today, the paradigm of water management is changing from focusing on draining to retaining as much water as possible. This is not an easy process; better understanding of the hydrological cycle can contribute to the necessary changes.

Keywords: climate change, water retention, convective precipitation, climate model, Synop observations
\newpage{}
\phantomsection
\stepcounter{articleid}
\addcontentsline{toc}{subsection}{Shi et~al. Enhancing Daily Precipitation Estimates: An Integrated Framework Using Near-Real-Time Satellite and Gauge Observations}
\begin{flushleft}

\abstrtitle{Enhancing Daily Precipitation Estimates: An Integrated Framework Using Near-Real-Time Satellite and Gauge Observations}

\name{Jiayong Shi$^1$, Juraj Parajka$^2$, Jianyun Zhang$^1$}

\index{Shi, Jiayong|textit}
\index{Parajka, Juraj}
\index{Zhang, Jianyun}
\institute{$^1$ Hohai University, College of Hydrology and Water Resources}

\institute{$^2$ Vienna University of Technology, Institute of Hydraulic Engineering and Water Resources Management}

\email{Corresponding author: Jiayong Shi, jyshihhu@163.com}

\end{flushleft}

\noindent

This study presents the EDGWR framework, a novel method for improving daily precipitation estimates by integrating satellite precipitation products (SPPs) with ground-based gauge data. Specifically, the research enhances the Integrated Multi-satellitE Retrievals for GPM Early Run (IMERG-E) product using gauge observations to addressing satellite data inaccuracies from sensor limitations and processing techniques. The methodology was tested using daily measurements from 12 meteorological stations in the Yellow River, China, source region from 2014 to 2018. Employing error decomposition techniques, the EDGWR framework identifies and corrects hit, missed, and false errors, using geographically weighted regression to increase the spatial accuracy of precipitation estimates. The results demonstrate that the enhanced IMERG-EDGWR product outperforms traditional methods like Inverse Distance Weighting (IDW) and Ordinary Kriging (OK), and standard IMERG products, particularly in capturing heavy and winter precipitation events. This framework significantly benefits hydrological modeling and weather forecasting, especially in regions with sparse rain gauge coverage thereby enhancing water resource management and mitigating natural disaster risks.

Keywords: satellite precipitation products; Precipitation fusion; Sparse gauge stations
\newpage{}
\phantomsection
\stepcounter{articleid}
\addcontentsline{toc}{subsection}{Kuban and Project. Climaax project: Climate risk assessments for every European region Heatwave toolbox (Heatwave toolbox)}
\begin{flushleft}

\abstrtitle{Climaax project: Climate risk assessments for every European region Heatwave toolbox (Heatwave toolbox)}

\name{Martin Kuban$^{1,2}$, Climaax Project$^3$}

\index{Kuban, Martin|textit}
\index{Project, Climaax}
\institute{$^1$ Slovak University of Technology, Department of Land and Water Resources Management}

\institute{$^2$ KAJO services}

\institute{$^3$ Project, created within the Climaax project}

\email{Corresponding author: martin.kuban@kajoservices.com}

\end{flushleft}

\noindent

The CLIMAte risk and vulnerability Assessment framework and toolbox (CLIMAAX) project is designed to empower local authorities with the tools needed to effectively adapt to climate change. As European regions face increasing risks due to climate-related hazards, municipalities often lack the resources and expertise necessary to implement resilient adaptation strategies. CLIMAAX addresses this challenge by providing an innovative set of toolboxes, which enable data-driven decision-making, vulnerability assessments, and the development of risk-informed adaptation plans.

This presentation will highlight the practical application of the CLIMAAX toolboxes, showcasing how these tools assist municipalities, such as those in Slovakia, in identifying localized climate risks, optimizing resource allocation, and fostering collaboration between scientific institutions and local stakeholders. The toolboxes offer tailored solutions for risk assessment, early warning systems, and scenario planning, ensuring that adaptation strategies are both scientifically robust and locally relevant.

The inclusion of a user-friendly interface and integration of real-time data enhances accessibility, making the CLIMAAX toolboxes vital for empowering cities to enhance climate resilience through data-driven, actionable insights.

Acknowledgements: This study was financially supported by the VEGA Grant Agency No. VEGA 1/0577/23.

Keywords: Climate, risk assessment, heatwave
\newpage{}
\phantomsection
\stepcounter{articleid}
\addcontentsline{toc}{subsection}{Katona et~al. Water balance calculations as a function of the soil water holding capacity in the area of Pannonhalma, Hungary}
\begin{flushleft}

\abstrtitle{Water balance calculations as a function of the soil water holding capacity in the area of Pannonhalma, Hungary}

\name{Máté Katona$^1$, Péter Végh$^1$, Adrienn Horváth$^1$, András Bidló$^1$}

\index{Katona, Máté|textit}
\index{Végh, Péter}
\index{Horváth, Adrienn}
\index{Bidló, András}
\institute{$^1$ University of Sopron, the Institute of Environment and Nature Conservation }

\email{Corresponding author: katona.mate@phd.uni-sopron.hu}

\end{flushleft}

\noindent

The water holding capacity of soils is an important factor in assessing the drought sensitivity of an area and determining its water management properties. However, in domestic forest site assessments, it is often overlooked in favor of estimations based on vegetation. In this study, we investigated the water management properties of forest soils in the Pannonhalma area by comparing them to current estimation functions. Soil samples were taken from different forest stands and analysed in a laboratory for their physical and chemical properties. Based on these measurements, several water-holding models were constructed to assess the soil water holding capacity. The results of these models can be used for forest management planning and to further soil physics research specific to forest soils. 

The present publication was supported by the NRDIF of the Ministry of Innovation and Technology (successor: Ministry of Culture and Innovation) under Project No. TKP2021-NKTA-43, funded by the TKP2021-NKTA grant programme. Project No. EKÖP-24-3-I has been implemented with support provided by the Ministry of Culture and Innovation of Hungary from the National Research, Development and Innovation Fund, financed under the EKÖP-24-3-I-SOE-30 funding scheme.

Keywords: water holding capacity, pedotransfer functions, water balance models
\newpage{}
\phantomsection
\stepcounter{articleid}
\addcontentsline{toc}{subsection}{Bertola. Record-breaking floods in Europe – are they surprising?}
\begin{flushleft}

\abstrtitle{Record-breaking floods in Europe – are they surprising?}

\name{Miriam Bertola$^1$}

\index{Bertola, Miriam|textit}
\institute{$^1$ Technische Universität Wien, Institute of Hydraulic Engineering and Water Resources Management}

\email{Corresponding author: bertola@hydro.tuwien.ac.at}

\end{flushleft}

\noindent

Record-breaking floods that far exceed previously observed records at a given location (i.e., megafloods) can take citizens and flood managers by surprise. In this study we analyse the most comprehensive dataset available to date of annual maximum discharges in Europe, to assess whether recent locally surprising megafloods could have been anticipated using observations in hydrologically similar catchments across the continent. We identify about 500 “target” catchments where recent megafloods have occurred that are surprising based on the local data. We perform a hindcast experiment of predicting their peak discharge with regional envelope curves, using flood observations from similar “donor” catchments up to the year before their occurrence. From this group of donor catchments, we construct an envelope curve which we compare with megafloods that occurred later in the target catchments. 

Our analysis shows that, in 95.5\% of the target catchments, the discharge of the envelope is larger than that of the megafloods observed, suggesting that, from a European perspective, almost none of the events can be considered a regional surprise.

Keywords: European floods; Record-breaking floods; Regional envelope curves
\newpage{}
\phantomsection
\stepcounter{articleid}
\addcontentsline{toc}{subsection}{Keve et~al. Investigation the relocation of landmines by flood events}
\begin{flushleft}

\abstrtitle{Investigation the relocation of landmines by flood events}

\name{Gábor Keve$^1$, Balázs Abonyi$^1$, Sándor Krikovszky$^1$, Dávid Csátaljay$^1$, Zoltán Kovács$^2$, István Ember$^2$, József Padányi$^2$}

\index{Keve, Gábor|textit}
\index{Abonyi, Balázs}
\index{Krikovszky, Sándor}
\index{Csátaljay, Dávid}
\index{Kovács, Zoltán}
\index{Ember, István}
\index{Padányi, József}
\institute{$^1$ Faculty of Water Sciences, Ludovika University of Public Service, Hungary}

\institute{$^2$ Faculty of Military Sciences and Officer Training, Ludovika University of Public Service, Hungary}

\email{Corresponding author: keve.gabor@uni-nke.hu}

\end{flushleft}

\noindent

Our research focuses on a study of the hydrodynamic behaviour of infantry landmines. Unfortunately, the cost of mine clearance is significantly higher than the cost of their deployment. The clearance work is particularly difficult if it is installed on flood plains where floodwaves can significantly reorder the original siting. 

Our study focused on two main experiments, which were conducted in a hydraulic flume with a cross-section of $50\cdot{}50$ cm and a length of 7\,m. The first experiment was designed to investigate the displacement of the landmines in the flume under the effect of different water velocities. Based on the experience of the first experiment, we further investigated the use of only one type of mine (PFM-1), which was placed at different positions in the flume with an artificial glass bottom and set at a 3\% slope, where the limiting velocities associated with the directions of the attack were determined. The water velocities required for the displacements were recorded during the experiments.

As a result of our research, we can provide preliminary estimates of the water velocities required to move landmines in grassland areas under different topographic conditions and directions of attack. Further experiments are planned, which will consider different terrain covers and types of landmines. 

Keywords: landmines, relocation of objects by floods
\newpage{}
\phantomsection
\stepcounter{articleid}
\addcontentsline{toc}{subsection}{Alivio et~al. How beneficial is urban tree planting for reducing stormwater runoff?}
\begin{flushleft}

\abstrtitle{How beneficial is urban tree planting for reducing stormwater runoff?}

\name{Mark Bryan Alivio$^1$, Nejc Bezak$^1$, Mojca Šraj$^1$, Matej Radinja$^1$}

\index{Alivio, Mark|textit}
\index{Bezak, Nejc}
\index{Sraj, Mojca@\v{S}raj, Mojca}
\index{Radinja, Matej}
\institute{$^1$ University of Ljubljana, Faculty of Civil and Geodetic Engineering, Ljubljana, Slovenia}

\email{Corresponding author: malivio@fgg.uni-lj.si}

\end{flushleft}

\noindent

As a vital element of urban “greening”, trees are praised for offering numerous social and
environmental benefits, making them a quintessential nature-based solution for more sustainable
cities. However, less emphasis is directed towards using the hydrological functions of trees in terms of
urban stormwater management, so this benefit is often underutilized. For urban areas with high
proportions of impervious surfaces, increasing the percentage of tree canopy cover and green space is
crucial in restoring the natural functioning of the ecosystem and water cycle. Urban tree canopies
provide an important pathway for stormwater management by routing the rainfall into various
components of the hydrological cycle. However, the representation of tree canopy hydrological
processes in most conventional stormwater models remains inadequate, often implicitly lumping the
specific parameters of individual tree species as part of vegetation in general or pervious accounting
process, rather than being explicitly represented. This study modelled and evaluated the stormwater
runoff reduction potential of birch (\textit{Betula pendula}) and pine (\textit{Pinus nigra}) trees in three scenarios
(i.e., birch, pine, and mixed-species planting) on a storm event basis using the updated Storm Water
Management Model (SWMM) tree canopy module. The interception routine implemented in the
updated SWMM model effectively captured the temporal evolution of throughfall and stemflow for
both trees in different phenoseasons. A strong correlation was found between the simulated and
observed throughfall ($r = 0.97-0.99$) and interception values ($r = 0.72$) across all the storm events. The
event-based results revealed that the reduction in runoff volume and peak flow across all the scenarios
and phenoseasons is between 20-25\% and 16-25\%, respectively. The mixed-species tree planting
scenario performed better in reducing both runoff volume and peak flow than the single-specie
scenarios. At the subcatchment level, the event analysis showed that incorporating these trees in
parking lots significantly reduces the runoff coefficient by 21.6-74.7\% during the leafed period and
30.8-77.3\% in the leafless period, depending on specific scenarios, storm events, and the number of
trees. However, the efficiency of stormwater reduction of both trees becomes limited during intense,
high-volume storm events, although they continue to provide tangible benefits. The water balance
analysis further emphasizes the contribution of canopy interception in the runoff reduction benefits of
urban trees, particularly during the leafed season, small to moderate storm events, and when trees are
in directly connected impervious areas. Infiltration and storage under the tree canopies play the
primary role in managing throughfall before it contributes to runoff, accounting for over 20\% of the
water balance. This study shows that the efficiency of stormwater reduction of urban trees depends on
both above and below-canopy processes.

Acknowledgment: This work was supported by the P2-0180 research program: “Water Science and
Technology, and Geotechnical Engineering: Tools and Methods for Process Analyses and Simulations,
and Development of Technologies” through the Ph.D. grant to the first author, which is financially
supported by the Slovenian Research and Innovation Agency (ARIS). Moreover, this study was also
carried out within the scope of the ongoing research projects J6-4628, J2-4489, and N2-0313
supported by the ARIS and SpongeScapes project (Grant Agreement ID No. 101112738) and
NATURE-DEMO (Grant Agreement IDNo. 101157448), which is supported by the European Union’s
Horizon Europe research and innovation programme.

Keywords: birch; pine; heavy rainfall; rainfall interception; runoff; stormwater; SWMM; tree canopy;
urban greening; urban trees
\newpage{}
\phantomsection
\stepcounter{articleid}
\addcontentsline{toc}{subsection}{Marjanović et~al. Quantifying the connectivity of saturated areas on an agricultural hillslope}
\begin{flushleft}

\abstrtitle{Quantifying the connectivity of saturated areas on an agricultural hillslope}

\name{Dušan Marjanović$^1$, Juraj Parajka$^1$, Borbala Szeles$^1$, Camillo Ress$^1$, Peter Strauß$^1$, Günter Blöschl$^1$}

\index{Marjanović, Dušan|textit}
\index{Parajka, Juraj}
\index{Szeles, Borbala}
\index{Ress, Camillo}
\index{Strauß, Peter}
\index{Blöschl, Günter}
\institute{$^1$ Vienna University of Technology, Institute of Hydraulic Engineering and Water Resources Management}

\institute{$^2$ Vienna University of Technology, Department of Geodesy and Geoinformation}

\institute{$^3$ Federal Agency for Water Management, Institute for Land and Water Management Research}

\email{Corresponding author: marjanovic@hydro.tuwien.ac.at}

\end{flushleft}

\noindent

Surface runoff from agricultural hillslopes is one of the most important factors controlling soil erosion, land degradation and stream water contamination. Understanding the connectivity of flow paths and their different properties is necessary for a more complete understanding of the system’s behaviour.

This contribution is based on the structural connectivity index scale and aims to present an estimation and evaluation of the connectivity of a saturated area on an agricultural hillslope using time-lapse photography. The study is conducted on a hillslope at the Hydrological Open Air Laboratory (HOAL) experimental catchment located in Petzenkirchen, Lower Austria. Using digital camera observations, the temporal dynamics of connectivity are estimated from the time-lapse photography and combined with observations of the runoff in the stream as well as hydrological variables such as precipitation, soil moisture, and groundwater levels.

Based on the structural connectivity scale index, the results allow for the identification of two groups of types of connectivity in the HOAL. The different types of connectivity exhibit hydrologically distinct behaviour. The formation of the two types of connectivity is mainly controlled by the initial soil moisture and rainfall characteristics.

Acknowledgment: This contribution is part of the ongoing research project entitled “Evaluation of the impact of rainfall interception on soil erosion” supported by the Slovenian Research and Innovation Agency (J2-4489) and the Austrian Science Fund (FWF) I 6254-N.

Keywords: connectivity, surface runoff, HOAL, time-lapse photography
\newpage{}
\phantomsection
\stepcounter{articleid}
\addcontentsline{toc}{subsection}{Herceg et~al. A Merriam-type canopy interception model developed for the European beech, considering its dynamic storage capacity}
\begin{flushleft}

\abstrtitle{A Merriam-type canopy interception model developed for the European beech, considering its dynamic storage capacity}

\name{András Herceg$^1$, Péter Kalicz$^1$, Katalin Zagyvai-Kiss$^1$, Katarina Zabret$^2$, Zoltán Gribovszki$^1$}

\index{Herceg, András|textit}
\index{Kalicz, Péter}
\index{Zagyvai-Kiss, Katalin}
\index{Zabret, Katarina}
\index{Gribovszki, Zoltán}
\institute{$^1$ University of Sopron, Faculty of Forestry, Institute of Geomatics and Civil Engineering}

\institute{$^2$ University of Ljubljana Faculty of Civil and Geodetic Engineering, Institute for Geo- and Hydro-Threats}

\email{Corresponding author: herceg.andras@uni-sopron.hu }

\end{flushleft}

\noindent

A comprehensive knowledge of rainfall distribution processes by tree canopies in the forest hydrological cycle is essential for understanding the ecosystem of the hydrology of forests.

Canopy interception is a time-varying item of the water balance, as the canopy storage capacity follows a seasonal pattern with a ratio of 13\% to the gross precipitation in the dormancy and 24\% in the growing season in the case of the European beech (Fagus sylvatica L.).

Considering the above-mentioned facts, the main objective of the present study was to develop a canopy regression interception model for the European beech partly based on its physical parameters, which takes into account the seasonally varying dynamic storage capacity of the canopy (with the application of remotely sensed LAI data). The model was tested with annual precipitation sums for the period 2017-2022.

The model’s results showed the rainfall distributional effect of the interception, which is significant in terms of its proportions (53\%), especially for small precipitation events (0-5 mm precipitation).

Acknowledgement
This article was made within the frame of Project No 143972SNN (OTKA) and TKP2021-NKTA-43. The TKP2021-NKTA-43 project which has been implemented with support provided by the Ministry of Innovation and Technology of Hungary (successor: Ministry of Culture and Innovation of Hungary) from the National Research, Development and Innovation Fund and financed under the TKP2021-NKTA funding scheme.

Keywords: interception, Merriam model, canopy storage capacity, beech, leaf area index
\newpage{}
