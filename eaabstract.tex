\phantomsection
\stepcounter{articleid}
\addcontentsline{toc}{subsection}{Szabó et~al. Groundwater monitoring in forests of the Great Hungarian Plain}
\begin{flushleft}

{\Large Solicited Speaker \bigskip}

\abstrtitle{Groundwater monitoring in forests of the Great Hungarian Plain}

\name{András Szabó$^1$, Ján Szolgay$^3$, Bence Bolla$^1$, Péter Kalicz$^2$, Zoltán Gribovszki$^2 $}

\index{Szabó, András|textit}
\index{Szolgay, Ján}
\index{Bolla, Bence}
\index{Kalicz, Péter}
\index{Gribovszki, Zoltán}
\institute{$^1$ Forest Research Institute, University of Sopron, Várkerület 30/A., 9600 Sárvár, Hungary.}

\institute{$^2$ Institute of Geomatics and Civil Engineering, Faculty of Forestry, University of Sopron, Bajcsy-Zsilinszky u. 4., 9400 Sopron, Hungary.}

\institute{$^3$ Department of Land and Water Resources Management Faculty of Civil Engineering, Slovak University of Technology, Radlinského 11, SK 81005 Bratislava, Slovakia.}

\email{szabo.andras@uni-sopron.hu}

\end{flushleft}

\noindent

In arid lowland areas, forests have a diverse ecological role, requiring significant water resources while also serving as important contributors to environmental cooling, soil moisture preservation, erosion control, and carbon storage. Consequently, significant efforts have been made over the past few decades to understand the complex interaction between forest vegetation and groundwater on the Great Hungarian Plain. With climate change altering meteorological conditions, the importance of drawing conclusions from measured data has increased. 

The hydro-meteorological monitoring system operated by the Forest Research Institute, University of Sopron has provided a basis for new findings in recent years, particularly regarding subsurface salt accumulation, groundwater uptake, diurnal fluctuations, recharge, and soil moisture. Looking ahead, this monitoring system also presents an opportunity for collaboration among experts from various fields.
This article was made in frame of the project TKP2021-NKTA-43 which has been implemented with the support provided by the Ministry of Innovation and Technology of Hungary (successor: Ministry of Culture and Innovation of Hungary) from the National Research, Development and Innovation Fund, financed under the TKP2021-NKTA funding scheme.

\includepdf[pages = -, addtotoc={1, subsubsection, 1, Short article,A01}]{ShortArt/O1Szabo.pdf}

\newpage{}
\phantomsection
\stepcounter{articleid}
\addcontentsline{toc}{subsection}{Báder et~al. Changes and trends in the climatic water balance of the Danube River basin based on meteorological, hydrological, and gravimetric data for the period 1961-2020}
\begin{flushleft}

\abstrtitle{Changes and trends in the climatic water balance of the Danube River basin based on meteorological, hydrological, and gravimetric data for the period 1961-2020}

\name{László Báder$^1$, József Szilágyi$^1$, Klaudia Négyesi$^1$, Eszter Nagy$^1$, Lóránt Földváry$^2$}

\index{Bader, Laszlo@B\'ader, L\'aszl\'o|textit}
\index{Szilágyi, József}
\index{Negyesi, Klaudia@N\'egyesi, Klaudia}
\index{Nagy, Eszter}
\index{Földváry, Lóránt}
\institute{$^1$ Department of Hydraulic and Water Resources Engineering, Faculty of Civil Engineering, Budapest University of Technology and Economics, H-1111 Műegyetem rkp. 1. Budapest, Hungary}

\institute{$^2$ Department of Geodesy and Surveying, Faculty of Civil Engineering, Budapest University of Technology and Economics, H-1111 Műegyetem rkp. 1. Budapest, Hungary}

\email{laszlo.bader@edu.bme.hu}

\end{flushleft}

\noindent

ERA5-Land gridded meteorological data for the Danube watershed show a significant increasing trend with $p=0.05$ for the annual mean net radiation and temperature and a significantly decreasing trend for the relative humidity. These forcing parameters have also led to an increase in evaporation with a significant trend of 1.7 mm yr$^{-1}$. The spatial distribution of the evaporation trends shows that lower-lying areas of central and eastern Europe are unable to meet the growing demand for evaporative water. Precipitation is decreasing in these areas.

The water loss in the Danube River basin, which is estimated as the annual decrease of the difference between the precipitation and evaporation trends, is about 1km$^3$ yr$^{-1}$. The magnitude is confirmed by the Terrestrial Water Storage Anomalies (TWSA) detected with gravimetry, which show a decreasing trend of 7 mm yr$^{-1}$ in the equivalent water height.

In Hungary, both the evaporation trend estimated by the complementary relationship method and the Penman reference evaporation trend show a significantly increasing trend. The demand (reference evaporation) is increasing by 2.23 ± 0.3 mm yr$^{-1}$, while the supply (land evaporation) is increasing by 1.64 ± 0.2 mm yr$^{-1}$. The gap between them is widening, indicating increasing vulnerability to droughts.

Keywords:
drought, evaporation trends, loss of water, climatic energy balance, gravimetry, mass anomaly

\includepdf[pages = -, addtotoc={1, subsubsection, 1, Short article,A02}]{ShortArt/O2BaderL.pdf}

\newpage{}
\phantomsection
\stepcounter{articleid}
\addcontentsline{toc}{subsection}{Nagy. Can we calibrate a recursive baseflow separation filter relying on catchment response time parameters?}
\begin{flushleft}

\abstrtitle{Can we calibrate a recursive baseflow separation filter relying on catchment response time parameters?}

\name{Eszter D. Nagy$^1$}

\index{Nagy, Eszter|textit}
\institute{$^1$ Department of Hydraulic and Water Resources Engineering, Faculty of Civil Engineering, Budapest University of Technology and Economics, H-1111 Műegyetem rkp. 1. Budapest, Hungary}

\email{nagy.eszter@emk.bme.hu}

\end{flushleft}

\noindent

The physically meaningful separation of a baseflow from an observed streamflow time series is still an unresolved problem in hydrology. This study provides a novel method that builds upon catchment response time to formulate an optimization criterion for calibrating the Lyne-Hollick filter. The catchment’s response time parameters, including the time of concentration ($T_c$) and time to equilibrium ($T_e$), can be derived from the observed precipitation and discharge time series. A range for the ratio of these two has already been derived on a physical basis. In the approach presented, the parameter of the Lyne-Hollick filter was calibrated so that the ratio of $T_c$ and $T_e$ falls within a physically plausible range. The proposed method was tested in 25 Hungarian catchments, and the results were compared to those derived in a previous study. The Pearson correlation coefficient improved from 0.654 to 0.862 between $T_c$ and the lag time when applying the proposed calibration procedure.
\newpage{}
\phantomsection
\stepcounter{articleid}
\addcontentsline{toc}{subsection}{Chagas et~al. Exploring drought-rich and flood-rich periods in Brazil}
\begin{flushleft}

\abstrtitle{Exploring drought-rich and flood-rich periods in Brazil}

\name{Vinícius B. P. Chagas$^{1,2}$, Pedro L. B. Chaffe$^1$, Günter Blöschl$^2$}

\index{Chagas, Vinícius|textit}
\index{Chaffe, Pedro}
\index{Blöschl, Günter}
\institute{$^1$ Department of Sanitary and Environmental Engineering, Federal University of Santa Catarina, Florianopolis, Brazil.}

\institute{$^2$ Institute of Hydraulic Engineering and Water Resources Management, Vienna University of Technology, Vienna, Austria.}

\email{chagas@hydro.tuwien.ac.at}

\end{flushleft}

\noindent

Streamflow exhibits persistent decadal variability; however, it is unclear if the magnitude and spatial extent of these variabilities are symmetrical for droughts and floods. Here, we examine drought-rich and flood-rich periods in 319 streamflow gauges in Brazil from 1940 to 2020. We contrast streamflow temporal clustering with rainfall, water abstraction, the Atlantic Multidecadal Oscillation (AMO), and the Pacific Decadal Oscillation (PDO). We found drought-rich periods in 61\% of the basins, which were 11.5 times the false positive rate and 2.2 times the flood-rich periods (28\%). The higher number of basins with drought-rich periods is linked with increased water abstractions since the 1990s, an interannual persistence of rainfall deficits, and a higher prevalence of rainfall-poor periods (44\% of the gauges) compared to rainfall-rich periods (29\% of the gauges). The drought-rich periods are aligned with the rainfall-poor periods, AMO, PDO, and increased water abstractions. These findings highlight the nonlinearity and asymmetry in changes in droughts and floods on decadal scales.
\newpage{}
\phantomsection
\stepcounter{articleid}
\addcontentsline{toc}{subsection}{Ehrendorfer et~al. Coupled snow-hydrological modelling for two high alpine Austrian catchments}
\begin{flushleft}

\abstrtitle{Coupled snow-hydrological modelling for two high alpine Austrian catchments}

\name{Caroline Ehrendorfer$^1$, Franziska Koch$^1$, Sophie Lücking$^1$, Thomas Pulka$^1$, Hubert Holzmann$^1$, Philipp Maier$^2$, Fabian Lehner$^2$, Herbert Formayer$^2$, Mathew Herrnegger$^1$}

\index{Ehrendorfer, Caroline|textit}
\index{Koch, Franziska}
\index{Lücking, Sophie}
\index{Pulka, Thomas}
\index{Holzmann, Hubert}
\index{Maier, Philipp}
\index{Lehner, Fabian}
\index{Formayer, Herbert}
\index{Herrnegger, Mathew}
\institute{$^1$ Institute of Hydrology and Water Management, Department of Water, Atmosphere and Environment, University of Natural Resources and Life Sciences, Vienna, Austria}

\institute{$^2$ Institute of Meteorology and Climatology, Department of Water, Atmosphere and Environment, University of Natural Resources and Life Sciences, Vienna, Austria}

\email{caroline.ehrendorfer@boku.ac.at}

\end{flushleft}

\noindent

The timing and quantity of snow and ice melt in high-alpine regions is of great importance, especially for time-sensitive processes such as hydropower production. In most conceptual hydrological models, the simulations of these components are frequently only based on simple temperature index methods. Additionally, the estimation of precipitation inputs for areas with complex terrain is characterised by a high degree of uncertainty.

This contribution aims to improve the simulation of relevant components of the water balance by coupling the conceptual rainfall-runoff model COSERO with the physically-based snowpack model Alpine3D. Two high-alpine catchments with intense hydropower activity are used as a case study, with hourly data for the period 2003-2020. Snow and ice processes are modelled by Alpine3D, while runoff routing and subsurface flow are computed in COSERO. Additional information from local weather stations and topography is integrated into the gridded meteorological input data. An iterative process is used to find local optimized precipitation correction factors. The proposed detailed and coupled simulation of glaciers, snowpack and runoff together with complex interpolations of meteorological inputs shows an improved simulation of reservoir inflow and discharge, especially during the snowmelt period.

Acknowledgements: We thank the VERBUND Energy4Business GmbH for fruitful discussions and providing us with data. This work was carried out as part of the HyMELT-CC project.
\newpage{}
\phantomsection
\stepcounter{articleid}
\addcontentsline{toc}{subsection}{Pölz et~al. Forecasting karst spring discharges using an interpretable machine learning approach}
\begin{flushleft}

\abstrtitle{Forecasting karst spring discharges using an interpretable machine learning approach}

\name{Anna Pölz$^{1,4}$, Alfred Paul Blaschke$^{1,4}$, Katalin Demeter$^{2,4}$, Andreas H. Farnleitner$^{2,3,4}$, Julia Derx$^{1,4}$}

\index{Pölz, Anna|textit}
\index{Blaschke, Alfred}
\index{Demeter, Katalin}
\index{Farnleitner, Andreas}
\index{Derx, Julia}
\institute{$^1$ Institute of Hydraulic Engineering and Water Resources Management, TU Wien, Vienna, Austria}

\institute{$^2$ Institute of Chemical, Environmental and Bioscience Engineering, Research Group Microbiology and Molecular Diagnostics, TU Wien, Vienna, Austria}

\institute{$^3$ Division Water Quality and Health, Karl Landsteiner University for Health Sciences, Krems, Austria}

\institute{$^4$ Interuniversity Cooperation Centre Water and Health, Austria }

\email{poelz@hydro.tuwien.ac.at}

\end{flushleft}

\noindent

Karst springs supply approximately 10\% of the global population with drinking water; understanding them therefore plays a vital role in water resource management. This study is aimed at forecasting karst spring discharges using machine learning models. A secondary aim was to increase the transparency of the forecast by providing attributions of the input variables. 

We have compared the performance of four machine learning models of varying complexities, a multivariate adaptive regression spline (MARS), a feed-forward neural network (ANN), a Long Short-Term Memory model (LSTM), and a Transformer (TF) to a Naïve baseline model for a highly dynamic Austrian alpine karst spring. Moreover, we have employed the DeepSHAP model explainability method to investigate the contributions of the input variables to each forecast and analyze the seasonal differences.

Our findings demonstrate that the complex models, i.e., the Transformer and LSTM, exhibit the best performances in forecasting the discharge, as quantified by the four evaluation metrics. The contributions of the input variables differ for each season. The input variables, discharge, UV-absorption, water temperature, air temperature and conductivity contributed most to the forecasts. 

The spring discharge forecasts, together with the model explanations, enhance understanding of the karst systems and aid in short-term water supply management and decision-making in karst regions.
\newpage{}
\phantomsection
\stepcounter{articleid}
\addcontentsline{toc}{subsection}{Tanhapour et~al. Investigating the impact of post-processing ensemble precipitation forecasts to simulate reservoir inflow}
\begin{flushleft}

\abstrtitle{Investigating the impact of post-processing ensemble precipitation forecasts to simulate reservoir inflow}

\name{Mitra Tanhapour$^{1,4}$, Jaber Soltani$^1$, Bahram Malekmohammadi$^2$, Hadi Shakibian$^3$, Kamila Hlavčová$^4$, Silvia Kohnová$^4$}

\index{Tanhapour, Mitra|textit}
\index{Soltani, Jaber}
\index{Malekmohammadi, Bahram}
\index{Shakibian, Hadi}
\index{Hlavčová, Kamila}
\index{Kohnová, Silvia}
\institute{$^1$ Water Engineering Department, Faculty of Agricultural Technology, University College of Agriculture \& Natural Resources, University of Tehran, Tehran, Iran}

\institute{$^2$ Department of Environmental Planning and Management, Graduate Faculty of Environment, University of Tehran, Tehran, Iran.}

\institute{$^3$ Department of Computer Engineering, Faculty of Engineering, Alzahra University, Tehran, Iran.}

\institute{$^4$ Department of Land and Water Resources Management, Faculty of Civil Engineering, Slovak University of Technology, Bratislava, Slovakia.}

\email{jsoltani@ut.ac.ir}

\end{flushleft}

\noindent

Flood forecasting mainly relies on the quality of precipitation forecasts. Numerical Weather Prediction (NWP) models play a crucial role in improving the reliability of precipitation forecasts. The direct application of ensemble precipitation forecasts derived from NWP models produces significant biases in hydrological modelling. Therefore, it is necessary to post-process ensemble precipitation forecasts. The aim of this research was post-processing the raw precipitation forecasts of three NWP models, i.e., UKMO, NCEP, and KMA, for six storm events that resulted in heavy floods in the Dez River basin, Iran, during 2013-2019. Post-processing of the ensemble precipitation forecasts was done in two steps. First, the regression models were used to correct the raw output of every individual NWP model. The second step developed a multi-model ensemble system using the Weighted Average‒Weighted Least Square Regression (WA-WLSR) model and the Group Method of Data Handling (GMDH) model. Then, the HBV hydrological model was fed with the post-processed ensemble precipitation forecasts to simulate the reservoir inflow floods. Besides, in order to investigate the effects of post-processing the ensemble precipitation forecasts, the ensemble inflow forecasts were compared with the deterministic inflow forecasts. The results demonstrated that the forecasting skill of the NWP models was improved using both the GMDH and WA-WLSR models, but the WA-WLSR model provided better results. The results of the HBV model in producing ensemble inflow forecasts showed that the normalized root mean square error (NRMSE) index decreased by an average of 21.3\%, compared to the deterministic inflow forecasts in the validation stage. This research methodology is practical in adopting more efficient strategies for the flood control management of reservoirs due to applying the uncertainty of inflow forecasts.

Keywords: Numerical weather prediction models; Multi-model ensemble system; Ensemble inflow forecasts; Deterministic inflow forecasts

Acknowledgments: This study was supported by the VEGA Grant Agency under Project No. 1/0782/21. The authors are very grateful for their research support.
\newpage{}
\phantomsection
\stepcounter{articleid}
\addcontentsline{toc}{subsection}{Štefunková. The impact of climate change on the aquatic habitat of the Hybica River}
\begin{flushleft}

\abstrtitle{The impact of climate change on the aquatic habitat of the Hybica River}

\name{Zuzana Štefunková$^1$}

\index{Stefunková, Zuzana@\v{S}tefunková, Zuzana|textit}
\institute{$^1$ Department of Land and Water Resources Management, Faculty of Civil Engineering Slovak University of Technology in Bratislava, Radlinského 11, 813 68 Bratislava, Slovakia}

\email{zuzana.stefunkova@stuba.sk}

\end{flushleft}

\noindent

The effects of climate change could affect the whole spectrum of the ecosystem. It disrupts the system of bonds, changes energy-material cycles, and thus disrupts the standard operation of a landscape, especially its components. Heavy rain and other frequent weather events are often repeated. They could lead to floods and reduced water quality, but also to a deterioration in the availability of water resources in some areas. One of the most determinative changes in weather extremes that we can expect is a protraction of drought seasons and the more frequent occurrence of minimum flows. This change in runoff conditions could have a direct impact on the ecosystem of a stream. Therefore, this study is aimed at determining the impact of climate change on an aquatic habitat in future decades. This problematic will be documented on the specific reach of the Hybica River.

The modelling of the quality of the aquatic habitat based on bio-indications was executed using IFIM methodology. Modelling based on bio-indications is a demanding procedure that is unavailable in design practice, so the results from dozens of streams were generalized. The analysis shows that generalized results can be used for mountain and piedmont streams.

The results of the modelling of the aquatic quality show that climate change may cause significant modifications in the ecosystem of watercourses. Based on these results, it is possible to design and evaluate restorative measures that could mitigate the impact of climate change on the in-stream areas of watercourses.

This study has been jointly supported by the Scientific Grant Agency under Contracts no. VEGA 1/0067/23.

\newpage{}
\phantomsection
\stepcounter{articleid}
\addcontentsline{toc}{subsection}{Škrinár. Study of the functioning of the fish passage on the Hron River in Slovakia.}
\begin{flushleft}

\abstrtitle{Study of the functioning of the fish passage on the Hron River in Slovakia.}

\name{Andrej Škrinár$^1$}

\index{Skrinár, Andrej@\v{S}krinár, Andrej|textit}
\institute{$^1$ Department of Land and Water Resources Management, Faculty of Civil Engineering, Slovak University of Technology}

\email{andrej.skrinar@stuba.sk}

\end{flushleft}

\noindent

River ecosystems can be considered the most diversified and productive ecosystems on the planet. Since the last century, the free flow of water, which ensures the basic functioning of lotic ecosystems and the constant dynamics of an environment, has been systematically disrupted by the construction of dams worldwide, thereby causing the massive loss of fish abundance and even biodiversity. Almost one in three freshwater species are globally threatened with extinction, with all taxonomic groups showing a higher risk of extinction in freshwater, compared to the terrestrial system.

On the way to their natural spawning habitats in the upper Hron River in southern Slovakia, from hundreds to thousands of fish, mostly barbels (Barbus barbus), have died in the turbines of the small hydro power plant of the Želiezovce reservoir, even though the weir has a implemented bio-corridor. Using 3-year ichthyological and hydraulic monitoring and robust 2D mathematical flow modelling, this study seeks to find the cause of this ecological disaster in order to prevent the recurrence of such events in the future.  

\newpage{}
